\documentclass{beamer}

\pdfmapfile{+sansmathaccent.map}


\mode<presentation>
{
	\usetheme{Warsaw} % or try Darmstadt, Madrid, Warsaw, Rochester, CambridgeUS, ...
	\usecolortheme{seahorse} % or try seahorse, beaver, crane, wolverine, ...
	\usefonttheme{serif}  % or try serif, structurebold, ...
	\setbeamertemplate{navigation symbols}{}
	\setbeamertemplate{caption}[numbered]
} 


%%%%%%%%%%%%%%%%%%%%%%%%%%%%
% itemize settings


%%%%%%%%%%%%%%%%%%%%%%%%%%%%
% itemize settings

\definecolor{myhotpink}{RGB}{255, 80, 200}
\definecolor{mywarmpink}{RGB}{255, 60, 160}
\definecolor{mylightpink}{RGB}{255, 80, 200}
\definecolor{mypink}{RGB}{255, 30, 80}
\definecolor{mydarkpink}{RGB}{155, 25, 60}

\definecolor{mypaleblue}{RGB}{240, 240, 255}
\definecolor{mylightblue}{RGB}{120, 150, 255}
\definecolor{myblue}{RGB}{90, 90, 255}
\definecolor{mygblue}{RGB}{70, 110, 240}
\definecolor{mydarkblue}{RGB}{0, 0, 180}
\definecolor{myblackblue}{RGB}{40, 40, 120}

\definecolor{mygreen}{RGB}{0, 200, 0}
\definecolor{mygreen2}{RGB}{245, 255, 230}

\definecolor{mygray}{gray}{0.8}
\definecolor{mydarkgray}{RGB}{80, 80, 160}

\definecolor{mydarkred}{RGB}{160, 30, 30}
\definecolor{mylightred}{RGB}{255, 150, 150}
\definecolor{myred}{RGB}{200, 110, 110}
\definecolor{myblackred}{RGB}{120, 40, 40}

\definecolor{mygreen}{RGB}{0, 200, 0}
\definecolor{mygreen2}{RGB}{205, 255, 200}

\definecolor{mydarkcolor}{RGB}{60, 25, 155}
\definecolor{mylightcolor}{RGB}{130, 180, 250}

\setbeamertemplate{itemize items}[default]

\setbeamertemplate{itemize item}{\color{myblackblue}$\blacksquare$}
\setbeamertemplate{itemize subitem}{\color{mygblue}$\blacktriangleright$}
\setbeamertemplate{itemize subsubitem}{\color{mygray}$\blacksquare$}

\setbeamercolor{palette quaternary}{fg=white,bg=mydarkgray}
\setbeamercolor{titlelike}{parent=palette quaternary}

\setbeamercolor{palette quaternary2}{fg=black,bg=mypaleblue}
\setbeamercolor{frametitle}{parent=palette quaternary2}

\setbeamerfont{frametitle}{size=\Large,series=\scshape}
\setbeamerfont{framesubtitle}{size=\normalsize,series=\upshape}





%%%%%%%%%%%%%%%%%%%%%%%%%%%%
% block settings

\setbeamercolor{block title}{bg=red!30,fg=black}

\setbeamercolor*{block title example}{bg=mygreen!40!white,fg=black}

\setbeamercolor*{block body example}{fg= black, bg= mygreen2}


%%%%%%%%%%%%%%%%%%%%%%%%%%%%
% URL settings
\hypersetup{
	colorlinks=true,
	linkcolor=blue,
	filecolor=blue,      
	urlcolor=blue,
}

%%%%%%%%%%%%%%%%%%%%%%%%%%

\renewcommand{\familydefault}{\rmdefault}

\usepackage{amsmath}
\usepackage{mathtools}

\usepackage{subcaption}

\usepackage{qrcode}

\DeclareMathOperator*{\argmin}{arg\,min}
\newcommand{\bo}[1] {\mathbf{#1}}

\newcommand{\R}{\mathbb{R}} 
\newcommand{\T}{^\top}     

\newcommand{\dx}[1] {\dot{\mathbf{#1}}}
\newcommand{\ma}[4] {\begin{bmatrix}
		#1 & #2 \\ #3 & #4
\end{bmatrix}}
\newcommand{\myvec}[2] {\begin{bmatrix}
		#1 \\ #2
\end{bmatrix}}
\newcommand{\myvecT}[2] {\begin{bmatrix}
		#1 & #2
\end{bmatrix}}


\newcommand{\mydate}{Spring 2023}

\newcommand{\mygit}{\textcolor{blue}{\href{https://github.com/SergeiSa/Control-Theory-Slides-Spring-2023}{github.com/SergeiSa/Control-Theory-Slides-Spring-2023}}}

\newcommand{\myqr}{ \textcolor{black}{\qrcode[height=1.5in]{https://github.com/SergeiSa/Control-Theory-Slides-Spring-2023}}
}

\newcommand{\myqrframe}{
	\begin{frame}
		\centerline{Lecture slides are available via Github, links are on Moodle}
		\bigskip
		\centerline{You can help improve these slides at:}
		\centerline{\mygit}
		\bigskip
		\myqr
	\end{frame}
}


\newcommand{\bref}[2] {\textcolor{blue}{\href{#1}{#2}}}

%%%%%%%%%%%%%%%%%%%%%%%%%%%%
% code settings

\usepackage{listings}
\usepackage{color}
% \definecolor{mygreen}{rgb}{0,0.6,0}
% \definecolor{mygray}{rgb}{0.5,0.5,0.5}
\definecolor{mymauve}{rgb}{0.58,0,0.82}
\lstset{ 
	backgroundcolor=\color{white},   % choose the background color; you must add \usepackage{color} or \usepackage{xcolor}; should come as last argument
	basicstyle=\footnotesize,        % the size of the fonts that are used for the code
	breakatwhitespace=false,         % sets if automatic breaks should only happen at whitespace
	breaklines=true,                 % sets automatic line breaking
	captionpos=b,                    % sets the caption-position to bottom
	commentstyle=\color{mygreen},    % comment style
	deletekeywords={...},            % if you want to delete keywords from the given language
	escapeinside={\%*}{*)},          % if you want to add LaTeX within your code
	extendedchars=true,              % lets you use non-ASCII characters; for 8-bits encodings only, does not work with UTF-8
	firstnumber=0000,                % start line enumeration with line 0000
	frame=single,	                   % adds a frame around the code
	keepspaces=true,                 % keeps spaces in text, useful for keeping indentation of code (possibly needs columns=flexible)
	keywordstyle=\color{blue},       % keyword style
	language=Octave,                 % the language of the code
	morekeywords={*,...},            % if you want to add more keywords to the set
	numbers=left,                    % where to put the line-numbers; possible values are (none, left, right)
	numbersep=5pt,                   % how far the line-numbers are from the code
	numberstyle=\tiny\color{mygray}, % the style that is used for the line-numbers
	rulecolor=\color{black},         % if not set, the frame-color may be changed on line-breaks within not-black text (e.g. comments (green here))
	showspaces=false,                % show spaces everywhere adding particular underscores; it overrides 'showstringspaces'
	showstringspaces=false,          % underline spaces within strings only
	showtabs=false,                  % show tabs within strings adding particular underscores
	stepnumber=2,                    % the step between two line-numbers. If it's 1, each line will be numbered
	stringstyle=\color{mymauve},     % string literal style
	tabsize=2,	                   % sets default tabsize to 2 spaces
	title=\lstname                   % show the filename of files included with \lstinputlisting; also try caption instead of title
}


%%%%%%%%%%%%%%%%%%%%%%%%%%%%
% URL settings
\hypersetup{
	colorlinks=false,
	linkcolor=blue,
	filecolor=blue,      
	urlcolor=blue,
}

%%%%%%%%%%%%%%%%%%%%%%%%%%

%%%%%%%%%%%%%%%%%%%%%%%%%%%%
% tikz settings

\usepackage{tikz}
\tikzset{every picture/.style={line width=0.75pt}}


\title{Riccati eq., Linear Quadratic Regulator}
\subtitle{Control Theory, Lecture 7}
\author{by Sergei Savin}
\centering
\date{\mydate}



\begin{document}
\maketitle


\begin{frame}{Content}
\begin{itemize}
\item Hamilton-Jacobi-Bellman equation
\begin{itemize}
    \item Definitions
    \item Cost, optimal cost
    \item Differentiating optimal cost
\end{itemize}
\item Algebraic Riccati equation
\begin{itemize}
    \item HJB for LTI
    \item Linear Quadratic Regulator
    \item Numerical methods
\end{itemize}
\end{itemize}
\end{frame}

\begin{frame}{Control policy}
%\framesubtitle{Definitions}
\begin{flushleft}

Let us define dynamics:

\begin{equation}
    \dot {\bo{x}} = \bo{f} (\bo{x}, \bo{u})
\end{equation}
%
with initial conditions $\bo{x}(0)$. 

\bigskip

Additionally we define \emph{control policy} as:

\begin{equation}
    \bo{u} = \pi (\bo{x}, t)
\end{equation}

To connect with the previous ways we talked about control, we can say that choosing different control gains and different feed-forward control amounts to choosing a different control policy.

\end{flushleft}
\end{frame}




\begin{frame}{Cost, optimal cost}
%\framesubtitle{}
\begin{flushleft}

Let $J$ be an additive cost function:

\begin{equation}
J (\bo{x}_0, \pi (\bo{x}, t)) = \int_0^\infty g(\bo{x}, \bo{u}) dt
\end{equation}
%
where $g(\bo{x}, \bo{u})$ is instantaneous cost and $\bo{x}_0 = \bo{x}(0)$ is the initial conditions. Notice that $J$ depends on $\bo{x}_0$ rather than $\bo{x}(t)$, since initial conditions and control policy completely define the trajectory of the system $\bo{x}(t)$.


\bigskip

Let $J^*$ be the optimal (lowest possible) cost. In other words:

\begin{equation}
J^*(\bo{x}_0) = \underset{\pi}{\inf{}} J(\bo{x}_0, \pi (\bo{x}, t))
\end{equation}

Optimal cost is attained when optimal policy is attained: $\pi = \pi^*(\bo{x}, t)$

\end{flushleft}
\end{frame}





%\begin{frame}{Hamilton-Jacobi-Bellman equation}
%\framesubtitle{Differentiating optimal cost}
%\begin{flushleft}
%
%
%Since $J^*(\bo{x}_0)$ does not depend on $t$, its full derivative is zero:
%
%\begin{equation}
%\frac{d J^*(\bo{x}_0)}{dt} = 0
%\end{equation}
%
%At the same time, we can expand the full derivative as follows:
%
%\begin{equation}
%\frac{d J^*}{dt } = 
%\frac{\partial J^*}{\partial \bo{x}} \dot {\bo{x}} +
%\frac{\partial J^*}{\partial t} = 0
%\end{equation}
%
%\bigskip
%
%Observe that $\frac{\partial J^*}{\partial t} = g(\bo{x}, \bo{u})$, and $\dot {\bo{x}} = \bo{f} (\bo{x}, \bo{u})$. Therefore:
%
%\begin{equation}
%\frac{\partial J^*}{\partial \bo{x}} \bo{f} (\bo{x}, \bo{u}) +
%g(\bo{x}, \bo{u}) = 0
%\end{equation}
%
%\end{flushleft}
%\end{frame}




\begin{frame}{Hamilton-Jacobi-Bellman equation}
% \framesubtitle{HJB}
\begin{flushleft}

With this, we can formulate \emph{Hamilton-Jacobi-Bellman equation} (HJB):

\begin{equation}
\label{eq:HJB_0}
\underset{\bo{u}}{\min} \ 
\left[ 
g(\bo{x}, \bo{u}) + 
\frac{\partial J^*}{\partial \bo{x}} \bo{f} (\bo{x}, \bo{u}) 
\right] = 0
\end{equation}

%This can be loosely interpreted as follows: the value in square brackets is $\dot J(\bo{x}_0, \pi)$, which is equal to 0 when $\pi = \pi^*(\bo{x}, t)$, and is positive otherwise (in the small vicinity of $\pi^*$), as $J(\bo{x}_0, \pi^*)$ is smaller than any $J(\bo{x}_0, \pi), \ \pi^* \neq \pi$.

\bigskip


We can find control that delivers minimum to the function \eqref{eq:HJB_0}:

\begin{equation}
\bo{u}^* = \underset{\bo{u}}{\text{argmin}} \ 
\left[ 
g(\bo{x}, \bo{u}) + 
\frac{\partial J^*}{\partial \bo{x}} \bo{f} (\bo{x}, \bo{u}) \right] 
\end{equation}

\end{flushleft}
\end{frame}





\begin{frame}{Algebraic Riccati (LTI case), 1}
%\framesubtitle{HJB for LTI}
\begin{flushleft}

For LTI, dynamics is:
\begin{equation}
\dot {\bo{x}} = \bo{A}  \bo{x} + \bo{B} \bo{u}
\end{equation}

We can choose quadratic cost:
\begin{equation}
g(\mathbf  x, \mathbf  u) = 
\mathbf  x^\top \bo{Q} \bo{x} +
\mathbf  u^\top \bo{R} \bo{u} 
\end{equation}

Then HJB becomes:
\begin{equation}
\underset{\bo{u}}{\min} \ [ 
\bo{x}^\top \bo{Q} \bo{x} +
\bo{u}^\top \bo{R} \bo{u} + 
\frac{\partial J^*}{\partial \bo{x}} 
(\bo{A} \bo{x} + \bo{B} \bo{u})] = 0
\end{equation}
%
where $\bo{Q} = \bo{Q}^\top \geq 0 $ and $\bo{R} = \bo{R}^\top > 0$.

\end{flushleft}
\end{frame}



\begin{frame}{Algebraic Riccati (LTI case), 2}
%\framesubtitle{HJB for LTI, part 2}
\begin{flushleft}

There is a theorem that says that for LTI with quadratic cost, $J^*$ has the form:

\begin{equation}
J^* = \mathbf  x^\top \bo{S} \bo{x}
\end{equation}
%
where $\bo{S} = \bo{S}^\top \geq 0$.

\bigskip

Then HJB becomes:

\[
\underset{\bo{u}}{\min} \ 
\left [ 
\mathbf  x^\top \bo{Q} \bo{x} +
\mathbf  u^\top \bo{R} \bo{u}
+ 
\bo{x}^\top \bo{S}
(\bo{A} \bo{x} + \bo{B} \bo{u}) 
+ 
(\bo{A} \bo{x} + \bo{B} \bo{u})^\top
\bo{S} \bo{x}
\right ] = 0
\]

Simplifying, we get:

\[
\underset{\bo{u}}{\min} \ 
\left [ 
\bo{u}^\top \bo{R} \bo{u}
+ 
\bo{x}^\top (
\bo{Q} + \bo{S} \bo{A} + \bo{A}^\top \bo{S}
)\bo{x}
+ 
\bo{x}^\top \bo{S} \bo{B} \bo{u} 
+ \bo{u}^\top \bo{B}^\top \bo{S} \bo{x} 
\right ] = 0
\]

\end{flushleft}
\end{frame}


\begin{frame}{Linear Quadratic Regulator, 1}
%\framesubtitle{}
\begin{flushleft}


Finding partial derivative of the HJB with respect to $\bo{u}$ and setting it to zero (as it is an extreme point) we get:
\begin{equation}
2 \mathbf  u^\top \bo{R} + 
2 \bo{x}^\top \bo{S} \bo{B} = 0
\end{equation}

This expression can be transposed and $\mathbf  u$ separated:

\begin{equation}
\mathbf  u = 
-\bo{R}^{-1} \bo{B}^\top \bo{S} \bo{x}
\end{equation}

This is the desired control law. We can see that it is \emph{proportional}. We can re-write it as:

\begin{equation}
\mathbf  u = -\mathbf K \bo{x}
\end{equation}

where $\mathbf K = \bo{R}^{-1} \bo{B}^\top \bo{S}$ is the controller gain. This control law is called Linear Quadratic Regulator (LQR).

\end{flushleft}
\end{frame}





\begin{frame}{Algebraic Riccati (LTI case), 3}
% \framesubtitle{Algebraic Riccati}
\begin{flushleft}

Substituting found control law into the HJB, we find:
\begin{equation}
\begin{split}
\underset{\bo{u}}{\min} \ 
[ 
\bo{x}^\top (
\bo{Q} + \bo{S} \bo{A} + \bo{A}^\top \bo{S}
)\bo{x}
+
\bo{x}^\top \bo{S} \bo{B} \bo{R}^{-1} \bo{R} \mathbf  R^{-1} \bo{B}^\top \bo{S} \bo{x}
- \\
- 
\bo{x}^\top \bo{S} \bo{B} \bo{R}^{-1} \bo{B}^\top \bo{S} \bo{x}
- 
\bo{x}^\top\bo{S} \bo{B} \bo{R}^{-1} \bo{B}^\top \bo{S} \bo{x} 
] = 0
\end{split}
\end{equation}

Simplifying, we get: 

\begin{equation}
\bo{x}^\top (\bo{Q} + \bo{S} \bo{A} + \bo{A}^\top \bo{S}
- \bo{S} \bo{B} \bo{R}^{-1} \bo{B}^\top \bo{S}) \bo{x} = 0
\end{equation}
%
which would hold for all $\bo{x}$ iff:
%

\begin{equation}
\bo{Q} - \bo{S} \bo{B} \bo{R}^{-1} \bo{B}^\top \bo{S} 
 + \bo{S} \bo{A} + \bo{A}^\top \bo{S} = 0
\end{equation}

This is the \emph{Algebraic Riccati equation}.

\end{flushleft}
\end{frame}



\begin{frame}{LQR via software}
%\framesubtitle{Numerical methods}
\begin{flushleft}

There are a number of ways to solve LQR:

\bigskip

\begin{itemize}
    \item In MATLAB there is a function \texttt{[K,S,P] = lqr(A,B,Q,R), where P=eig(A-B*K)}
    \item In Python, there is \texttt{S = scipy.linalg.solve\_continuous\_are(A,B,Q,R)}
%    \item In Drake there is a function \texttt{(K,S) = LinearQuadraticRegulator(A,B,Q,R)}
\end{itemize}

\end{flushleft}
\end{frame}



\begin{frame}{LQR and pole placement}
%	\framesubtitle{Numerical methods}
	\begin{flushleft}
		
		\begin{itemize}
			\item Pole placement \textcolor{mydarkgreen}{upsides}: allows to design exactly how fast the control error decays to zero; allows to design control error oscillations.
			
			\item Pole placement \textcolor{red}{downsides}: may require unreasonably high control gains. Easy to ask for "unreasonable" performance.
			
			\item LQR \textcolor{mydarkgreen}{upsides}: easy to produce "reasonable" control gains.
			
			\item LQR \textcolor{red}{downsides}: may produce very slow decaying control error with oscillations.
		\end{itemize}
		
		
	\end{flushleft}
\end{frame}




\begin{frame}{Discrete case}
	%	\framesubtitle{Numerical methods}
	\begin{flushleft}
		
		Consider discrete dynamics:
		
		\begin{equation}
			\bo{x}_{i+1} = \bo{A}\bo{x}_i + \bo{B}\bo{u}_i
		\end{equation}
		
		with a cost function:
		
		\begin{equation}
			J = \sum_{i=0}^N (\bo{x}_i\T \bo{Q} \bo{x}_i +  \bo{u}_i\T \bo{R} \bo{u}_i)
		\end{equation}
		
		Let us find the optimal control policy for this case.
		
		
	\end{flushleft}
\end{frame}




\begin{frame}{Cost function}
	%	\framesubtitle{Numerical methods}
	\begin{flushleft}
		
		Let us consider a linear control law $\bo{u}_i = - \bo{K}_i \bo{x}_i$. Given initial conditions $\bo{x}_0$ we find $\bo{x}_1$ as:
		
		\begin{equation}
			\bo{x}_1 =(\bo{A} -  \bo{B}\bo{K}_i )\bo{x}_0
		\end{equation}
		
		let us define components of the cost associated with each iteration as $J_1$, $J_2$, etc:
		
		\begin{equation}
			J_i = \bo{x}_i\T \bo{Q} \bo{x}_i +  \bo{u}_i\T \bo{R} \bo{u}_i
		\end{equation}
		
		Since $\bo{u}_0 = - \bo{K} \bo{x}_0$ we can show that $J_1$ is a quadratic form of initial conditions:
		
		\begin{equation}
			J_0 = \bo{x}_0\T (\bo{Q}+   \bo{K}_i\T \bo{R} \bo{K}_i) \bo{x}_0
		\end{equation}
		
		Similarly, we can show that all $J_i$ can be written as a quadratic form of initial conditions.
		
	\end{flushleft}
\end{frame}




\begin{frame}{Cost-to-go, 1}
	%	\framesubtitle{Numerical methods}
	\begin{flushleft}
		
		Let us define cost-to-go as optimal cost for given initial conditions:
		%
		\begin{equation}
			V_0 = \underset{\bo{u}}{\text{min}} \sum_{i=0}^N (\bo{x}_i\T \bo{Q} \bo{x}_i +  \bo{u}_i\T \bo{R} \bo{u}_i)
		\end{equation}
		
		If $\bo{x}_0$, $\bo{x}_1$, $\bo{x}_2$, ... is a sequence of states that form an optimal trajectory, let us define the cost-to-go starting from each of these states as:
		%
		\begin{equation}
			V_i =  \underset{\bo{u}}{\text{min}} \sum_{k = i}^N (\bo{x}_k\T \bo{Q} \bo{x}_k +  \bo{u}_k\T \bo{R} \bo{u}_k)
		\end{equation}
		
		We can note that the optimal cost will take a form of a quadratic function:
		
		\begin{equation}
			V_i = \bo{x}_i\T \bo{P}_i \bo{x}_i
		\end{equation}
		
		
		
	\end{flushleft}
\end{frame}




\begin{frame}{Cost-to-go, 2}
	%	\framesubtitle{Numerical methods}
	\begin{flushleft}
		
		
		Defining $\bo{x}_0 = \bo{z}$ and $\bo{u}_0 = \bo{w}$ we can write cost-to-go as:
		
		\begin{equation}
			\label{Bellman}
			V_0(\bo{z}) = \underset{\bo{w}}{\text{min}} 
			(\bo{z}\T \bo{Q} \bo{z} +  \bo{w}\T \bo{R} \bo{w}  + V_1(\bo{A}\bo{z} + \bo{B}\bo{w})  )
		\end{equation}
		%
		where $V_1(\bo{A}\bo{z} + \bo{B}\bo{w})$ is the optimal cost-to-go on the next step. 
		
		\bigskip
		
		As the next step is closer to the goal, the optimal cost-to-go on the next step is both smaller than on the current step, and is contained in it.
		
		\bigskip
		
		The equation \eqref{Bellman} is called \emph{Bellman} equation.
		
	\end{flushleft}
\end{frame}



\begin{frame}{Discrete LQR, 1}
	%	\framesubtitle{Numerical methods}
	\begin{flushleft}
		
		Since $V_0 = \bo{x}_0\T \bo{P}_0 \bo{x}_0$ and $V_1 = \bo{x}_1\T \bo{P}_1 \bo{x}_1$ we can re-write Bellman equation as:
		
		\begin{equation}
			\bo{z}\T \bo{P}_0 \bo{z}
			 = \underset{\bo{w}}{\text{min}} 
			(\bo{z}\T \bo{Q} \bo{z} +  \bo{w}\T \bo{R} \bo{w}  + 
			(\bo{A}\bo{z} + \bo{B}\bo{w})\T \bo{P}_1 (\bo{A}\bo{z} + \bo{B}\bo{w}) 
			 )
		\end{equation}
		
		To find minimum over $\bo{w}$ we take partial derivative and set it to zero:
		
		\begin{align}
			\frac{\partial}{\partial \bo{w}}
			\left( 
			\bo{z}\T \bo{Q} \bo{z} +  \bo{w}\T \bo{R} \bo{w}  + 
			(\bo{A}\bo{z} + \bo{B}\bo{w})\T \bo{P}_1 (\bo{A}\bo{z} + \bo{B}\bo{w}) 
			\right) = 0
		\end{align}
		%
		\begin{align}
		2\bo{w}\T \bo{R}  + 
		2(\bo{A}\bo{z} + \bo{B}\bo{w})\T \bo{P}_1 \bo{B} = 0
		\end{align}
		%
		\begin{align}
			\bo{R}\bo{w} + \bo{B}\T \bo{P}_1 \bo{B} \bo{w} + \bo{B}\T \bo{P}_1 \bo{A}\bo{z} = 0
		\end{align}
		
	\end{flushleft}
\end{frame}




\begin{frame}{Discrete LQR, 2}
	%	\framesubtitle{Numerical methods}
	\begin{flushleft}
		
		From $\bo{R}\bo{w} + \bo{B}\T \bo{P}_1 \bo{B} \bo{w} + \bo{B}\T \bo{P}_1 \bo{A}\bo{z} = 0$, we can find expression for the optimal control law:
		
		\begin{align}
			(\bo{R} + \bo{B}\T \bo{P}_1 \bo{B}) \bo{w} = -\bo{B}\T \bo{P}_1 \bo{A}\bo{z}
			\\
			\bo{w} = -(\bo{R} + \bo{B}\T \bo{P}_1 \bo{B})^{-1} \bo{B}\T \bo{P}_1 \bo{A}\bo{z}
		\end{align}
		
		In general, given optimal cost-to-go matrix $\bo{P}_{i+1}$we find optimal control law:
		
		\begin{equation}
			\bo{u}_i = -(\bo{R} + \bo{B}\T \bo{P}_{i+1} \bo{B})^{-1} \bo{B}\T \bo{P}_{i+1} \bo{A}\bo{x}_i
		\end{equation}
		
		
	\end{flushleft}
\end{frame}






\begin{frame}{Back-propagation, 1}
	%	\framesubtitle{Numerical methods}
	\begin{flushleft}
		
		Let us define $\bo{M} = (\textcolor{myblue}{\bo{R} + \bo{B}\T \bo{P}_{i+1} \bo{B}})^{-1}$ and $\bo{N} =\textcolor{mydarkgreen}{\bo{B}\T \bo{P}_{i+1} \bo{A}}$ we can re-write the control law:
		%
		\begin{equation}
			\bo{u}_i = -\bo{M} \bo{N} \bo{x}_i
		\end{equation}
		
		We can substitute the control law into the Bellman eq.:
	%
		\begin{equation*}
			\bo{x}_i\T \bo{P}_i \bo{x}_i
			= \underset{\bo{u}}{\text{min}} 
			(\bo{x}_i\T \bo{Q} \bo{x}_i +  \textcolor{myred}{\bo{u}_i}\T \bo{R} \textcolor{myred}{\bo{u}_i}  + 
			(\bo{A}\bo{x}_i + \bo{B}\textcolor{myred}{\bo{u}_i})\T \bo{P}_{i+1} (\bo{A}\bo{x}_i + \bo{B}\textcolor{myred}{\bo{u}_i}) 
			)
		\end{equation*}
		%
		\begin{align*}
		\bo{x}_i\T \bo{P}_i \bo{x}_i
		= 
		\bo{x}_i\T \bo{Q} \bo{x}_i +  \textcolor{myred}{\bo{x}_i\T \bo{N}\T \bo{M}} \bo{R} \textcolor{myred}{\bo{M} \bo{N} \bo{x}_i}  + 
		\\
		+
		(\bo{A}\bo{x}_i - \bo{B}\textcolor{myred}{\bo{M} \bo{N} \bo{x}_i})\T \bo{P}_{i+1} (\bo{A}\bo{x}_i - \bo{B}\textcolor{myred}{\bo{M} \bo{N} \bo{x}_i}) 
		\end{align*}
		%
		\begin{align*}
		\bo{P}_i 
		= 
		\bo{Q} +  \bo{N}\T \bo{M}\textcolor{myblue}{ \bo{R}} \bo{M} \bo{N} + \bo{A}\T \bo{P}_{i+1} \bo{A} - 
		\textcolor{mydarkgreen}{\bo{A}\T \bo{P}_{i+1} \bo{B}} \bo{M} \bo{N} - \\
		 \bo{N}\T  \bo{M} \textcolor{mydarkgreen}{\bo{B}\T  \bo{P}_{i+1} \bo{A}}  +
		 \bo{N}\T  \bo{M}\textcolor{myblue}{ \bo{B}\T  \bo{P}_{i+1} \bo{B}} \bo{M} \bo{N} 
		\end{align*}
		%
		\begin{align*}\hspace*{-0.8cm} %\scriptstyle
		\bo{P}_i 
		= 
		\bo{Q}  + \bo{A}\T \bo{P}_{i+1} \bo{A}  +  \bo{N}\T \bo{M}\textcolor{myblue}{ (\bo{R} + \bo{B}\T  \bo{P}_{i+1} \bo{B})} \bo{M} \bo{N}
		-\bo{N}\T \bo{M} \bo{N} 
		-\bo{N}\T  \bo{M} \bo{N}  
		\end{align*}
		%
		\begin{align*}
		\bo{P}_i = 	\bo{Q} + \bo{A}\T \bo{P}_{i+1} \bo{A} - \bo{N}\T  \bo{M} \bo{N}  
		\end{align*}
		
		
		
	\end{flushleft}
\end{frame}




\begin{frame}{Back-propagation, 2}
	%	\framesubtitle{Numerical methods}
	\begin{flushleft}
		
		From $\bo{P}_i = 	\bo{Q} + \bo{A}\T \bo{P}_{i+1} \bo{A} - \textcolor{mydarkgreen}{\bo{N}}\T  \textcolor{myblue}{\bo{M}} \textcolor{mydarkgreen}{\bo{N}}  $ we obtain the final result:
		
		\begin{align*}
			\bo{P}_i = 	\bo{Q} + \bo{A}\T \bo{P}_{i+1} \bo{A} - 
			\textcolor{mydarkgreen}{\bo{A}\T\bo{P}_{i+1}\bo{B}}  \textcolor{myblue}{(\bo{R} + \bo{B}\T \bo{P}_{i+1} \bo{B})^{-1} } \textcolor{mydarkgreen}{\bo{B}\T \bo{P}_{i+1} \bo{A}}
		\end{align*}
		
		This equation can be used to compute $\bo{P}_i$ from known $\bo{P}_{i+1}$.
		
	\end{flushleft}
\end{frame}


\begin{frame}{Further reading}
	%	\framesubtitle{Numerical methods}
	\begin{flushleft}
		
\begin{itemize}
	\item \bref{http://underactuated.mit.edu/lqr.html}{Underactuated robotics. Linear Quadratic Regulators}.
	
	\item \bref{https://stanford.edu/class/ee363/lectures/dlqr.pdf}{Discrete LQR. Stanford, EE363}.
	
	\item \bref{https://web.stanford.edu/class/ee363/lectures/dlqr-ss.pdf}{Discrete LQR (infinite horizon). Stanford, EE363}.
\end{itemize}
		
		
	\end{flushleft}
\end{frame}


\myqrframe


\begin{frame}
	\centerline{\huge Appendix A: Illustration of HJB}
\end{frame}


\begin{frame}{Optimality, definitions}
	%\framesubtitle{}
	\begin{flushleft}
		
		Consider the additive cost $J (\bo{x}_0, \pi (\bo{x})) = \int_0^\infty g(\bo{x}, \bo{u}) dt$, where $\bo{u} = \pi (\bo{x})$ is a control policy. The function $g(\bo{x}, \bo{u}) \geq 0$ can be interpreted as a rate of change of cost.
		
		\bigskip
		
		Let $\pi^*(\bo{x})$ be the optimal control policy. Applying the optimal policy to the dynamics $\dot{\bo{x}} = f(\bo{x}, \bo{u})$ we obtain optimal dynamics:
		%
		\begin{equation}
			\dot{\bo{x}} = f^*(\bo{x}) = f(\bo{x}, \pi^*(\bo{x}))
		\end{equation}
		
		Given initial conditions $\bo{x}_0 = \bo{z}$ we generate an optimal trajectory $\bo{x}^* = \bo{x}^*(t, \bo{z})$. Given optimal trajectory and optimal control policy we find optimal cost:
		%
		\begin{equation}
			J^*(\bo{z}) = J(\bo{z}, \pi^* (\bo{x}))
		\end{equation}
		
		Equivalently, we find optimal instantenious cost:
		%
		\begin{equation}
			g^*(\bo{x}) = g(\bo{x}, \pi^* (\bo{x}))
		\end{equation}
		
		
	\end{flushleft}
\end{frame}



\begin{frame}{Incurred cost}
	%\framesubtitle{}
	\begin{flushleft}
		
		Since optimal cost depends on initial consitions only, we can find a function $J^* = J^*(\bo{z})$ defined over $\R^n$.
		
		\bigskip
		
		Lets us consider a trajectory $\bo{x}^* = \bo{x}^*(t, \bo{z})$ and sequential points on this trajectory $\bo{x}_0$, $\bo{x}_1$, $\bo{x}_2$, etc. associated with the time points $t_0$, $t_1$, $t_2$, etc. We can define incurred cost (incurred while moving from the initial state $\bo{z} = \bo{x}_0$ to the given point) for each of these points: $S_0$, $S_1$, $S_2$, etc.:
		
		\begin{equation}
			S_i = \int_0^{t_i} g^*(\bo{x}) dt
		\end{equation}
		
		Since $g^*(\bo{x}) \geq 0$, we observe that $S_0 \leq S_1 \leq S_2 \leq ...$. Moving along a trajectory $\bo{x}^*(t, \bo{z})$ we incur monotonically increasing cost. We can describe it as a time function $S(t)$. The rate of increace of this function is given by instantenious cost $g^*(\bo{x})$.
		
		
	\end{flushleft}
\end{frame}



\begin{frame}{Cost-to-go}
	%\framesubtitle{}
	\begin{flushleft}
		
		We know that the optimal cost from the point $\bo{z}$ is given as $J^*(\bo{z})$. For each sequential point on the trajectory we can define \emph{cost-to-go} $V_i$ as a difference between the optimal cost and the incurred cost:
		
		\begin{equation}
			V_i = J^*(\bo{z}) - S_i
		\end{equation}
		
		For a given trajectory, we can describe cost-to-go as a time function $V(t) = J^*(\bo{z}) - S(t)$. Where as $S(t)$ is monotonically increasing, the function $V(t)$ is monotonically decreasing, with a rate of change given as $-g^*(\bo{x})$.
		
		\bigskip
		
		Note that the cost-to-go can be equivalently found as:
		
		\begin{equation}
			V(t) = J^*(\bo{x}^*(t))
		\end{equation}
		%
		since the optimal cost we incur by starting from the point $\bo{x}_i$ is equivalent to the cost "have left to incur" when we reach $\bo{x}_i$ from $\bo{x}_0$.
		
	\end{flushleft}
\end{frame}


\begin{frame}{Optimality}
	%\framesubtitle{}
	\begin{flushleft}
		
		With that, we can make an observation: for an optimal policy we see the rate of change of the cost-to-go function equal to $-g^*(\bo{x})$. But this rate of change can be found by taking a derivative of $J^*(\bo{x})$ with respect to the vector field $\dot{\bo{x}} = f^*(\bo{x})$:
		
		\begin{equation}
			-g^*(\bo{x}) = \frac{\partial J^*}{\partial \bo{x}} f^*(\bo{x})
		\end{equation}
		
		For sub-optimal control policies, the incurred cost will outpace the decrease of the cost-to-go:
		
		\begin{equation}
			g(\bo{x}, \bo{u}) + \frac{\partial J^*}{\partial \bo{x}} f(\bo{x}, \bo{u}) \geq 0
		\end{equation}
		
		Optimal policy recovers the sought equality:
		
		\begin{equation}
			\underset{\bo{u}}{\min} \ 
			\left[ 
			g(\bo{x}, \bo{u}) + 
			\frac{\partial J^*}{\partial \bo{x}} \bo{f} (\bo{x}, \bo{u}) 
			\right] = 0
		\end{equation}
		
	\end{flushleft}
\end{frame}





\end{document}
