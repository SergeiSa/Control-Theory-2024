\documentclass{beamer}

\pdfmapfile{+sansmathaccent.map}


\mode<presentation>
{
	\usetheme{Warsaw} % or try Darmstadt, Madrid, Warsaw, Rochester, CambridgeUS, ...
	\usecolortheme{seahorse} % or try seahorse, beaver, crane, wolverine, ...
	\usefonttheme{serif}  % or try serif, structurebold, ...
	\setbeamertemplate{navigation symbols}{}
	\setbeamertemplate{caption}[numbered]
} 


%%%%%%%%%%%%%%%%%%%%%%%%%%%%
% itemize settings


%%%%%%%%%%%%%%%%%%%%%%%%%%%%
% itemize settings

\definecolor{myhotpink}{RGB}{255, 80, 200}
\definecolor{mywarmpink}{RGB}{255, 60, 160}
\definecolor{mylightpink}{RGB}{255, 80, 200}
\definecolor{mypink}{RGB}{255, 30, 80}
\definecolor{mydarkpink}{RGB}{155, 25, 60}

\definecolor{mypaleblue}{RGB}{240, 240, 255}
\definecolor{mylightblue}{RGB}{120, 150, 255}
\definecolor{myblue}{RGB}{90, 90, 255}
\definecolor{mygblue}{RGB}{70, 110, 240}
\definecolor{mydarkblue}{RGB}{0, 0, 180}
\definecolor{myblackblue}{RGB}{40, 40, 120}

\definecolor{mygreen}{RGB}{0, 200, 0}
\definecolor{mygreen2}{RGB}{245, 255, 230}

\definecolor{mygray}{gray}{0.8}
\definecolor{mydarkgray}{RGB}{80, 80, 160}

\definecolor{mydarkred}{RGB}{160, 30, 30}
\definecolor{mylightred}{RGB}{255, 150, 150}
\definecolor{myred}{RGB}{200, 110, 110}
\definecolor{myblackred}{RGB}{120, 40, 40}

\definecolor{mygreen}{RGB}{0, 200, 0}
\definecolor{mygreen2}{RGB}{205, 255, 200}

\definecolor{mydarkcolor}{RGB}{60, 25, 155}
\definecolor{mylightcolor}{RGB}{130, 180, 250}

\setbeamertemplate{itemize items}[default]

\setbeamertemplate{itemize item}{\color{myblackblue}$\blacksquare$}
\setbeamertemplate{itemize subitem}{\color{mygblue}$\blacktriangleright$}
\setbeamertemplate{itemize subsubitem}{\color{mygray}$\blacksquare$}

\setbeamercolor{palette quaternary}{fg=white,bg=mydarkgray}
\setbeamercolor{titlelike}{parent=palette quaternary}

\setbeamercolor{palette quaternary2}{fg=black,bg=mypaleblue}
\setbeamercolor{frametitle}{parent=palette quaternary2}

\setbeamerfont{frametitle}{size=\Large,series=\scshape}
\setbeamerfont{framesubtitle}{size=\normalsize,series=\upshape}





%%%%%%%%%%%%%%%%%%%%%%%%%%%%
% block settings

\setbeamercolor{block title}{bg=red!30,fg=black}

\setbeamercolor*{block title example}{bg=mygreen!40!white,fg=black}

\setbeamercolor*{block body example}{fg= black, bg= mygreen2}


%%%%%%%%%%%%%%%%%%%%%%%%%%%%
% URL settings
\hypersetup{
	colorlinks=true,
	linkcolor=blue,
	filecolor=blue,      
	urlcolor=blue,
}

%%%%%%%%%%%%%%%%%%%%%%%%%%

\renewcommand{\familydefault}{\rmdefault}

\usepackage{amsmath}
\usepackage{mathtools}

\usepackage{subcaption}

\usepackage{qrcode}

\DeclareMathOperator*{\argmin}{arg\,min}
\newcommand{\bo}[1] {\mathbf{#1}}

\newcommand{\R}{\mathbb{R}} 
\newcommand{\T}{^\top}     

\newcommand{\dx}[1] {\dot{\mathbf{#1}}}
\newcommand{\ma}[4] {\begin{bmatrix}
		#1 & #2 \\ #3 & #4
\end{bmatrix}}
\newcommand{\myvec}[2] {\begin{bmatrix}
		#1 \\ #2
\end{bmatrix}}
\newcommand{\myvecT}[2] {\begin{bmatrix}
		#1 & #2
\end{bmatrix}}


\newcommand{\mydate}{Spring 2023}

\newcommand{\mygit}{\textcolor{blue}{\href{https://github.com/SergeiSa/Control-Theory-Slides-Spring-2023}{github.com/SergeiSa/Control-Theory-Slides-Spring-2023}}}

\newcommand{\myqr}{ \textcolor{black}{\qrcode[height=1.5in]{https://github.com/SergeiSa/Control-Theory-Slides-Spring-2023}}
}

\newcommand{\myqrframe}{
	\begin{frame}
		\centerline{Lecture slides are available via Github, links are on Moodle}
		\bigskip
		\centerline{You can help improve these slides at:}
		\centerline{\mygit}
		\bigskip
		\myqr
	\end{frame}
}


\newcommand{\bref}[2] {\textcolor{blue}{\href{#1}{#2}}}

%%%%%%%%%%%%%%%%%%%%%%%%%%%%
% code settings

\usepackage{listings}
\usepackage{color}
% \definecolor{mygreen}{rgb}{0,0.6,0}
% \definecolor{mygray}{rgb}{0.5,0.5,0.5}
\definecolor{mymauve}{rgb}{0.58,0,0.82}
\lstset{ 
	backgroundcolor=\color{white},   % choose the background color; you must add \usepackage{color} or \usepackage{xcolor}; should come as last argument
	basicstyle=\footnotesize,        % the size of the fonts that are used for the code
	breakatwhitespace=false,         % sets if automatic breaks should only happen at whitespace
	breaklines=true,                 % sets automatic line breaking
	captionpos=b,                    % sets the caption-position to bottom
	commentstyle=\color{mygreen},    % comment style
	deletekeywords={...},            % if you want to delete keywords from the given language
	escapeinside={\%*}{*)},          % if you want to add LaTeX within your code
	extendedchars=true,              % lets you use non-ASCII characters; for 8-bits encodings only, does not work with UTF-8
	firstnumber=0000,                % start line enumeration with line 0000
	frame=single,	                   % adds a frame around the code
	keepspaces=true,                 % keeps spaces in text, useful for keeping indentation of code (possibly needs columns=flexible)
	keywordstyle=\color{blue},       % keyword style
	language=Octave,                 % the language of the code
	morekeywords={*,...},            % if you want to add more keywords to the set
	numbers=left,                    % where to put the line-numbers; possible values are (none, left, right)
	numbersep=5pt,                   % how far the line-numbers are from the code
	numberstyle=\tiny\color{mygray}, % the style that is used for the line-numbers
	rulecolor=\color{black},         % if not set, the frame-color may be changed on line-breaks within not-black text (e.g. comments (green here))
	showspaces=false,                % show spaces everywhere adding particular underscores; it overrides 'showstringspaces'
	showstringspaces=false,          % underline spaces within strings only
	showtabs=false,                  % show tabs within strings adding particular underscores
	stepnumber=2,                    % the step between two line-numbers. If it's 1, each line will be numbered
	stringstyle=\color{mymauve},     % string literal style
	tabsize=2,	                   % sets default tabsize to 2 spaces
	title=\lstname                   % show the filename of files included with \lstinputlisting; also try caption instead of title
}


%%%%%%%%%%%%%%%%%%%%%%%%%%%%
% URL settings
\hypersetup{
	colorlinks=false,
	linkcolor=blue,
	filecolor=blue,      
	urlcolor=blue,
}

%%%%%%%%%%%%%%%%%%%%%%%%%%

%%%%%%%%%%%%%%%%%%%%%%%%%%%%
% tikz settings

\usepackage{tikz}
\tikzset{every picture/.style={line width=0.75pt}}


\title{Dynamic Controllers, Filters}
\subtitle{Control Theory, Lecture 9}
\author{by Sergei Savin}
\centering
\date{\mydate}



\begin{document}
\maketitle



\begin{frame}{Content}
\begin{itemize}
\item Systems - static, dynamic
\item Dynaic Controller
\item PID
\item Filters
\end{itemize}
\end{frame}




\begin{frame}{Systems}
%\framesubtitle{How do we know the state?}
\begin{flushleft}

A system is characterized by how it relates its input to its output.

\begin{itemize}
	\item Controller $\bo{u} = \bo{K}\bo{x}$ is a system that relates $\bo{x}$ - its input - to $\bo{u}$, its output. 
	
	\item Plant (in state-space representation) 
$\begin{cases}
		\dot {\bo{x}} = \bo{A} \bo{x} + \bo{B} \bo{u} \\
		\bo{y} = \bo{C} \bo{x}
	\end{cases}$ 
relates its input $\bo{u}$ to its output $\bo{y}$.

	\item Plant (in Laplace representation) $Y(s) = W(s)X(s)$ with a transfer function $W(s)$ is a system relating its input signal $X(s)$ to its output signal $Y(s)$.
\end{itemize}

\end{flushleft}
\end{frame}



\begin{frame}{Linear systems, static}
	%\framesubtitle{How do we know the state?}
	\begin{flushleft}
		
		In this lecture we consider linear systems only. 
		
		\bigskip
		
		For some systems, the relation between input and output does not depend on time. Such system we can call \emph{static}.
		
		\bigskip
		
		Examples of static systems:
		
		\begin{itemize}
			\item (State-Space) controller $\bo{u} = \bo{K}\bo{x}$. 
			
			\item (Laplace) gain function $Y(s) = 10 X(s)$.
		\end{itemize}
		
	\end{flushleft}
\end{frame}


\begin{frame}{Linear systems, dynamic}
	%\framesubtitle{How do we know the state?}
	\begin{flushleft}
		
		Alternatively, there are linear systems for which relation between input and output depends on time. Such system we can call \emph{dynamic}.
		
		\bigskip
		
		Examples of dynamic systems:
		
		\begin{itemize}
			\item (State-Space) plant $\begin{cases}
				\dot {\bo{x}} = \bo{A} \bo{x} + \bo{B} \bo{u} \\
				\bo{y} = \bo{C} \bo{x}
			\end{cases}$.
			
			\item (Laplace) plant $Y(s) = \frac{1}{s^2 + 2s + 7} X(s)$.
			
			\item (ODE) plant $\ddot y + 5 \dot y + y = u$.
		\end{itemize}
		
		\bigskip
		
		For these systems, the output depends not only on the current (time-wise) value of the input, but on the entire history of input values.
		
	\end{flushleft}
\end{frame}



\begin{frame}{Static vs dynamic}
	%\framesubtitle{How do we know the state?}
	\begin{flushleft}
		
		We can think of static systems as a form of algebraic linear equations, while dynamic systems are linear differential equations.
		
		\bigskip
		
		The dintinguishing quality of dynamic systems is that they have \emph{state}. We can think of state as an internal variable which changes with time, affecting the relation between the input and the output of the system.
		
	\end{flushleft}
\end{frame}



\begin{frame}{Controller+Observer}
	%\framesubtitle{How do we know the state?}
	\begin{flushleft}
		
		Controller + Luenberger observer is also a dynamical system with input $\bo{y}$, output $\bo{u}$ and state $\hat{\bo{x}}$:
		
		\begin{equation}
			\begin{cases}
				\hat{\dot {\bo{x}}} = \bo{A} \hat{\bo{x}} + \bo{B} \mathbf u + \bo{L}(\mathbf y - \bo{C} \hat{\bo{x}})
				\\
				\bo{u} = -\bo{K} \hat{\bo{x}}
			\end{cases}
		\end{equation}
		
		The form of this system resebles the plant with output equation.
		
		\bigskip
		
		We can think of this system as a \emph{dynamic controller} (rather then separating it into a dynamic observer and a static controller). 
		
	\end{flushleft}
\end{frame}




\begin{frame}{Dynamic controller}
	%\framesubtitle{How do we know the state?}
	\begin{flushleft}
		
		A generic form of dynamic controller is:
		
		\begin{equation}
			\begin{cases}
				\hat{\dot {\bo{x}}} = \bo{A}_k \hat{\bo{x}} + \bo{B}_k \bo{y}
				\\
				\bo{u} = \bo{C}_k \hat{\bo{x}} + \bo{D}_k \bo{y}
			\end{cases}
		\end{equation}
		
		where matrices $\bo{A}_k$, $\bo{B}_k$, $\bo{C}_k$ and $\bo{D}_k$ can be tuned to achieve better performance.
		
		\bigskip
		
		A Luenberger observer + static controller is a particular instance of dynamic controller.
		
	\end{flushleft}
\end{frame}



\begin{frame}{PID, 1}
	%\framesubtitle{How do we know the state?}
	\begin{flushleft}
		
		One of the better known examples of dynamic controlllers is a \emph{proportional-integral-derivative controller (PID)}. Scaler case:
		
		\begin{equation}
			u(t) = k_p e(t) + k_d \frac{d}{dt} e(t) + k_i \int_{0}^{t} e(\tau) d \tau
		\end{equation}
		
		Vector case:
		
		\begin{equation}
			\bo{u}(t) = \bo{K}_p \bo{e}(t) + \bo{K}_d \frac{d}{dt} \bo{e}(t) + \bo{K}_i \int_{0}^{t} \bo{e}(\tau) d \tau
		\end{equation}
		
		This controller works for second-order systems (in $\bo{e}(t)$) where $\bo{e}$ and $\dot{\bo{e}}$ are the state of the system.
		
		\bigskip
		
		This controller is equivalent to the linear controllers we studied in the course (e.g. LQR) but with integral part allowing for \emph{robustness} - the ability to compensate for static additive disturbance in the model.
		
	\end{flushleft}
\end{frame}



\begin{frame}{PID, 2}
	%\framesubtitle{How do we know the state?}
	\begin{flushleft}
		
		We can re-write PID in Laplace space:
		
		\begin{equation}
			U(s) = \left(k_p  + k_d s + k_i \frac{1}{s} \right )E(s)
		\end{equation}
		
		\begin{equation}
			U(s) =
			\frac{k_d s^2 + k_p s + k_i }{s}E(s)
		\end{equation}
		
		It is not instantly obvious that the controller has state.
		
	\end{flushleft}
\end{frame}




\begin{frame}{PID, 3}
	%\framesubtitle{How do we know the state?}
	\begin{flushleft}
		
		To re-write it in the state-space form, we introduce input to the controller (output to the plant) $\bo{y} = \begin{bmatrix}
		\bo{e} \\ \dot{\bo{e}}
		\end{bmatrix}$. We define a state $\bo{z} = \int_{0}^{t} \bo{e}(\tau) d \tau$, giving us the following state-space form of the controller:
		
		\begin{equation}
			\begin{cases}
				\dot{ \bo{z} }  = 
				\begin{bmatrix}
					\bo{I} & 0
				\end{bmatrix}\bo{y}
				\\
				\bo{u} = \bo{K}_i \bo{z} + 
				\begin{bmatrix}
					 \bo{K}_p &  \bo{K}_d
				\end{bmatrix} \bo{y}
			\end{cases}
		\end{equation}
		
	\end{flushleft}
\end{frame}



\begin{frame}{Filters, 1}
	%\framesubtitle{How do we know the state?}
	\begin{flushleft}
		
		Consider a system:
		
		\begin{equation}
			\begin{cases}
				\dot{ x }  = 
				ax+bu
				\\
				y = c x + d \sin(wt)
			\end{cases}
		\end{equation}
		
		where $w >>0$ is a frequency of the external disturbance and $d$ is it amplitute.
		
		\bigskip
		
		We can first pass the measurement $y$ through a system that attenuates high-frequency component of a signal; such system is called a \emph{low-pass filter} and then pass it through the controller:
		
		\begin{equation}
			\begin{cases}
				\dot{ z }  = 
				-z+y
				\\
				u = k z
			\end{cases}
		\end{equation}
		 
		
	\end{flushleft}
\end{frame}




\begin{frame}{Filters, 2}
	%\framesubtitle{How do we know the state?}
	\begin{flushleft}
		
		The filter $\dot{ z }  = 
		-z+y$ has a Laplace form:
		
		\begin{equation}
			Z(s) = \frac{1}{s+1} Y(s)
		\end{equation}
		
		Frequency responce of this transfer function is:
		
		\begin{equation}
			W(\omega) = \frac{1}{\sqrt{\omega^2 + 1}} Y(s)
		\end{equation}
		
		As frequency increases, the signal is attenuated more. For frequencies close to zero, the signal passes nearly unchanged.
		
	\end{flushleft}
\end{frame}


\myqrframe

\end{document}
