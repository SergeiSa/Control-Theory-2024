\documentclass{beamer}

\pdfmapfile{+sansmathaccent.map}


\mode<presentation>
{
	\usetheme{Warsaw} % or try Darmstadt, Madrid, Warsaw, Rochester, CambridgeUS, ...
	\usecolortheme{seahorse} % or try seahorse, beaver, crane, wolverine, ...
	\usefonttheme{serif}  % or try serif, structurebold, ...
	\setbeamertemplate{navigation symbols}{}
	\setbeamertemplate{caption}[numbered]
} 


%%%%%%%%%%%%%%%%%%%%%%%%%%%%
% itemize settings


%%%%%%%%%%%%%%%%%%%%%%%%%%%%
% itemize settings

\definecolor{myhotpink}{RGB}{255, 80, 200}
\definecolor{mywarmpink}{RGB}{255, 60, 160}
\definecolor{mylightpink}{RGB}{255, 80, 200}
\definecolor{mypink}{RGB}{255, 30, 80}
\definecolor{mydarkpink}{RGB}{155, 25, 60}

\definecolor{mypaleblue}{RGB}{240, 240, 255}
\definecolor{mylightblue}{RGB}{120, 150, 255}
\definecolor{myblue}{RGB}{90, 90, 255}
\definecolor{mygblue}{RGB}{70, 110, 240}
\definecolor{mydarkblue}{RGB}{0, 0, 180}
\definecolor{myblackblue}{RGB}{40, 40, 120}

\definecolor{mygreen}{RGB}{0, 200, 0}
\definecolor{mygreen2}{RGB}{245, 255, 230}

\definecolor{mygray}{gray}{0.8}
\definecolor{mydarkgray}{RGB}{80, 80, 160}

\definecolor{mydarkred}{RGB}{160, 30, 30}
\definecolor{mylightred}{RGB}{255, 150, 150}
\definecolor{myred}{RGB}{200, 110, 110}
\definecolor{myblackred}{RGB}{120, 40, 40}

\definecolor{mygreen}{RGB}{0, 200, 0}
\definecolor{mygreen2}{RGB}{205, 255, 200}

\definecolor{mydarkcolor}{RGB}{60, 25, 155}
\definecolor{mylightcolor}{RGB}{130, 180, 250}

\setbeamertemplate{itemize items}[default]

\setbeamertemplate{itemize item}{\color{myblackblue}$\blacksquare$}
\setbeamertemplate{itemize subitem}{\color{mygblue}$\blacktriangleright$}
\setbeamertemplate{itemize subsubitem}{\color{mygray}$\blacksquare$}

\setbeamercolor{palette quaternary}{fg=white,bg=mydarkgray}
\setbeamercolor{titlelike}{parent=palette quaternary}

\setbeamercolor{palette quaternary2}{fg=black,bg=mypaleblue}
\setbeamercolor{frametitle}{parent=palette quaternary2}

\setbeamerfont{frametitle}{size=\Large,series=\scshape}
\setbeamerfont{framesubtitle}{size=\normalsize,series=\upshape}





%%%%%%%%%%%%%%%%%%%%%%%%%%%%
% block settings

\setbeamercolor{block title}{bg=red!30,fg=black}

\setbeamercolor*{block title example}{bg=mygreen!40!white,fg=black}

\setbeamercolor*{block body example}{fg= black, bg= mygreen2}


%%%%%%%%%%%%%%%%%%%%%%%%%%%%
% URL settings
\hypersetup{
	colorlinks=true,
	linkcolor=blue,
	filecolor=blue,      
	urlcolor=blue,
}

%%%%%%%%%%%%%%%%%%%%%%%%%%

\renewcommand{\familydefault}{\rmdefault}

\usepackage{amsmath}
\usepackage{mathtools}

\usepackage{subcaption}

\usepackage{qrcode}

\DeclareMathOperator*{\argmin}{arg\,min}
\newcommand{\bo}[1] {\mathbf{#1}}

\newcommand{\R}{\mathbb{R}} 
\newcommand{\T}{^\top}     

\newcommand{\dx}[1] {\dot{\mathbf{#1}}}
\newcommand{\ma}[4] {\begin{bmatrix}
		#1 & #2 \\ #3 & #4
\end{bmatrix}}
\newcommand{\myvec}[2] {\begin{bmatrix}
		#1 \\ #2
\end{bmatrix}}
\newcommand{\myvecT}[2] {\begin{bmatrix}
		#1 & #2
\end{bmatrix}}


\newcommand{\mydate}{Spring 2023}

\newcommand{\mygit}{\textcolor{blue}{\href{https://github.com/SergeiSa/Control-Theory-Slides-Spring-2023}{github.com/SergeiSa/Control-Theory-Slides-Spring-2023}}}

\newcommand{\myqr}{ \textcolor{black}{\qrcode[height=1.5in]{https://github.com/SergeiSa/Control-Theory-Slides-Spring-2023}}
}

\newcommand{\myqrframe}{
	\begin{frame}
		\centerline{Lecture slides are available via Github, links are on Moodle}
		\bigskip
		\centerline{You can help improve these slides at:}
		\centerline{\mygit}
		\bigskip
		\myqr
	\end{frame}
}


\newcommand{\bref}[2] {\textcolor{blue}{\href{#1}{#2}}}

%%%%%%%%%%%%%%%%%%%%%%%%%%%%
% code settings

\usepackage{listings}
\usepackage{color}
% \definecolor{mygreen}{rgb}{0,0.6,0}
% \definecolor{mygray}{rgb}{0.5,0.5,0.5}
\definecolor{mymauve}{rgb}{0.58,0,0.82}
\lstset{ 
	backgroundcolor=\color{white},   % choose the background color; you must add \usepackage{color} or \usepackage{xcolor}; should come as last argument
	basicstyle=\footnotesize,        % the size of the fonts that are used for the code
	breakatwhitespace=false,         % sets if automatic breaks should only happen at whitespace
	breaklines=true,                 % sets automatic line breaking
	captionpos=b,                    % sets the caption-position to bottom
	commentstyle=\color{mygreen},    % comment style
	deletekeywords={...},            % if you want to delete keywords from the given language
	escapeinside={\%*}{*)},          % if you want to add LaTeX within your code
	extendedchars=true,              % lets you use non-ASCII characters; for 8-bits encodings only, does not work with UTF-8
	firstnumber=0000,                % start line enumeration with line 0000
	frame=single,	                   % adds a frame around the code
	keepspaces=true,                 % keeps spaces in text, useful for keeping indentation of code (possibly needs columns=flexible)
	keywordstyle=\color{blue},       % keyword style
	language=Octave,                 % the language of the code
	morekeywords={*,...},            % if you want to add more keywords to the set
	numbers=left,                    % where to put the line-numbers; possible values are (none, left, right)
	numbersep=5pt,                   % how far the line-numbers are from the code
	numberstyle=\tiny\color{mygray}, % the style that is used for the line-numbers
	rulecolor=\color{black},         % if not set, the frame-color may be changed on line-breaks within not-black text (e.g. comments (green here))
	showspaces=false,                % show spaces everywhere adding particular underscores; it overrides 'showstringspaces'
	showstringspaces=false,          % underline spaces within strings only
	showtabs=false,                  % show tabs within strings adding particular underscores
	stepnumber=2,                    % the step between two line-numbers. If it's 1, each line will be numbered
	stringstyle=\color{mymauve},     % string literal style
	tabsize=2,	                   % sets default tabsize to 2 spaces
	title=\lstname                   % show the filename of files included with \lstinputlisting; also try caption instead of title
}


%%%%%%%%%%%%%%%%%%%%%%%%%%%%
% URL settings
\hypersetup{
	colorlinks=false,
	linkcolor=blue,
	filecolor=blue,      
	urlcolor=blue,
}

%%%%%%%%%%%%%%%%%%%%%%%%%%

%%%%%%%%%%%%%%%%%%%%%%%%%%%%
% tikz settings

\usepackage{tikz}
\tikzset{every picture/.style={line width=0.75pt}}


\title{Controllability, Observability}
\subtitle{Control Theory, Lecture 9}
\author{by Sergei Savin}
\centering
\date{\mydate}



\begin{document}
\maketitle


\begin{frame}{Content}
\begin{itemize}
\item Cayley–Hamilton
\item Controllability of Discrete LTI
\begin{itemize}
    \item Controllability matrix
    \item Controllability criterion
\end{itemize}
\item Observability of Discrete LTI
\begin{itemize}
    \item Dual system
    \item Observability criterion
\end{itemize}
%\item Analytical solution to ODE
%\item Forced State Response
\item Controllability of Continuous-Time LTI
\end{itemize}
\end{frame}





\begin{frame}{Cayley–Hamilton}
	% \framesubtitle{Limited control}
	\begin{flushleft}
		
		Equation $\text{det}(\bo{M} - \lambda\bo{I}) = 0$ is called \emph{characteristic equation} of matrix $\bo{M}$, its roots being eigenvalues of the matrix.
		
		\bigskip
		
		\begin{theorem}[Cayley–Hamilton]
			A matrix $\bo{M} \in \R^{n, n}$ satisfies its own characteristic equation.
		\end{theorem}
		
		A characteristic equation can be written as $\lambda^n + a_{n-1}\lambda^{n-1} + ... + a_0  = 0$, meaning that we can write:
		
		\begin{equation}
			\bo{M}^n + a_{n-1}\bo{M}^{n-1} + ... + a_0\bo{I}  = 0
		\end{equation}	
		
		Meaning that \textcolor{mydarkblue}{$\bo{M}^n$ is a linear combination of $\bo{M}^{n-1}$, $\bo{M}^{n-2}$, ..., $\bo{I}$}.
		
		
	\end{flushleft}
\end{frame}


\begin{frame}{Definitions}
	% \framesubtitle{Limited control}
	\begin{flushleft}
		
		\begin{definition}[Controllability]
			A system is controllable on time interval $t_0 \leq t \leq t_f$, if it is possible to find control input $u(t)$ that would drive the system to a desired state $\bo{x}(t_f)$ from any initial state $\bo{x}(t_0)$.
		\end{definition}
		
		\begin{definition}[Observability]
			A system is observable on time interval $t_0 \leq t \leq t_f$, if using output $\bo{y}(t)$ on that time interval it is possible to estimate exactly the state of the system $\bo{x}(t_f)$, given any initial estimation error.
		\end{definition}
	
		\begin{definition}[Observability, alternative]
				A system is observable on time interval $t_0 \leq t \leq t_f$, if any initial state $\bo{x}(t_0)$ is uniquely determined by output $\bo{y}(t)$ on that interval.
		\end{definition}
		
	\end{flushleft}
\end{frame}





\begin{frame}{Controllability of Discrete LTI}
% \framesubtitle{Definitions}
\begin{flushleft}

Consider discrete LTI:
\begin{equation}
\bo{x}_{i+1} = \bo{A}  \bo{x}_i + \bo{B} \bo{u}_i
\end{equation}

Assume the initial state is $\bo{x}_1$. Then we can deduce that:

\begin{align*}
\bo{x}_2 &= \bo{A} \bo{x}_1 + \bo{B} \bo{u}_1 \\
\bo{x}_3 &= \bo{A} \bo{x}_2 + \bo{B} \bo{u}_2 = \bo{A} (\bo{A} \bo{x}_1 + \bo{B} \bo{u}_1) + \bo{B} \bo{u}_2 \\
\bo{x}_4 &= \bo{A} \bo{x}_3 + \bo{B} \bo{u}_3 = \bo{A} (\bo{A} (\bo{A} \bo{x}_1 + \bo{B} \bo{u}_1) + \bo{B} \bo{u}_2) + \bo{B} \bo{u}_3 \\
... \\
\bo{x}_{n+1} &= \bo{A}^n \bo{x}_1 + \bo{A}^{n-1} \bo{B} \bo{u}_1 + ... + 
\bo{A}^{n - k} \bo{B} \bo{u}_{k} + ... + 
\bo{B} \bo{u}_n
\end{align*}

\end{flushleft}
\end{frame}



\begin{frame}{Controllability matrix}
%\framesubtitle{Controllability matrix}
\begin{flushleft}

Equation $\bo{x}_{n+1} = \bo{A}^n \bo{x}_1 + \bo{A}^{n-1} \bo{B} \bo{u}_1 + ... 
+ \bo{A}^{n - k} \bo{B} \bo{u}_{k} + ...
\bo{B} \bo{u}_n$ can be re-written as:

\begin{equation}
    \bo{x}_{n+1} - \bo{A}^n \bo{x}_1 = 
    \begin{bmatrix}
    \bo{B} &
    \bo{A} \bo{B} &
    \bo{A}^2 \bo{B} & ... &
    \bo{A}^{n - 1} \bo{B}
    \end{bmatrix}    
    \begin{bmatrix}
    \bo{u}_{n} \\
    \bo{u}_{n-1} \\
    \bo{u}_{n-2} \\ ... \\
    \bo{u}_{1}
    \end{bmatrix}
\end{equation}

Notice that in order for the system to go from $\bo{x}_1$ to $\bo{x}_{n+1}$, vector $\bo{x}_{n+1} - \bo{A}^n \bo{x}_1$ needs be in the column space of $\mathcal{C} = \begin{bmatrix}
    \bo{B} &
    \bo{A} \bo{B} & ... &
    \bo{A}^{n - 1} \bo{B}
    \end{bmatrix}$.

Since $\bo{x}_{n+1}$ can be anything, and $\bo{x}_1$ might be equal to zero (among other possibilities), we should require that all vectors in $\R^n$ are in the column space of $\mathcal{C}$, meaning $\mathcal{C}$ needs to be full row rank.

\end{flushleft}
\end{frame}


\begin{frame}{Controllability criterion}
%\framesubtitle{Controllability criterion}
\begin{flushleft}

\begin{block}{Controllability}
For a system $\bo{x}_{i+1} = \bo{A}  \bo{x}_i + \bo{B} \bo{u}_i$, where $\bo{x} \in \R^n$, if the matrix $\mathcal{C} = \begin{bmatrix}
    \bo{B} &
    \bo{A} \bo{B} & ... &
    \bo{A}^{n - 1} \bo{B}
    \end{bmatrix}$ is full row rank (i.e. $\text{rank}(\mathcal{C}) = n$), any state can be reached, which means that \emph{the system is controllable}.
\end{block}

\end{flushleft}
\end{frame}




\begin{frame}{Controllability matrix rank}
	% \framesubtitle{Limited control}
	\begin{flushleft}
		
		What happens if we add more columns to the controllability matrix, for example $\bo{A}^n \bo{B}$? Consider the matrix:
		
		\begin{equation}
			\mathcal{C}_+ = \begin{bmatrix}
				\bo{B} &
				\bo{A} \bo{B} & ... &
				\bo{A}^{n - 1} \bo{B} &
				\bo{A}^n \bo{B}
			\end{bmatrix}
		\end{equation}
		
		But from Cayley–Hamilton we know that: 
		
		\begin{align}
			\bo{A}^n = -a_{n-1}\bo{A}^{n-1} - ... - a_0\bo{I} 
			\\
			\bo{A}^n\bo{B} = -a_{n-1}\bo{A}^{n-1}\bo{B} - ... - a_0\bo{B}
		\end{align}
		
		Meaning that columns of $\bo{A}^n\bo{B}$ are expressed as linear combination of columns of $\mathcal{C}$, hence the matrix $\mathcal{C}_+$ has the same rank as $\mathcal{C}$.
		
	\end{flushleft}
\end{frame}


%\begin{frame}{Controllability matrix rank}
%	% \framesubtitle{Limited control}
%	\begin{flushleft}
%		
%		If you are interested why the controllability matrix for not include more columns, like $\bo{A}^n$, Consider the following: 
%		
%		The controllability matrix can be written as 
%		
%		\begin{equation}
%			\mathcal{C} = \begin{bmatrix}
%				\bo{I} &
%				\bo{A}  & ... &
%				\bo{A}^{n - 1} 
%			\end{bmatrix}
%			\begin{bmatrix}
%				\bo{B} & \bo{0} & ... & \bo{0} \\
%				\bo{0}  & \bo{B} & ... & \bo{0} \\
%				...  & ... & ... & ... \\
%				\bo{0}  & \bo{0} & ... & \bo{B} 
%			\end{bmatrix}
%		\end{equation}
%		
%		meaning that the rank of $\mathcal{C}$ depends only on matrix $\begin{bmatrix}
%			\bo{I} &
%			\bo{A}  & ... &
%			\bo{A}^{n - 1} 
%		\end{bmatrix}$. Adding  to it columns $\bo{A}^n$ does not change the rank, as $\bo{A}^n$ is a linear combination of the other columns, as we proved in the previous slide.
%		
%	\end{flushleft}
%\end{frame}




\begin{frame}{Observability of Discrete LTI}
% \framesubtitle{Definitions}
\begin{flushleft}

Consider discrete LTI:
\begin{equation}
\begin{cases}
\bo{x}_{i+1} = \bo{A}  \bo{x}_i + \bo{B} \bo{u}_i \\
\bo{y}_i     = \bo{C}  \bo{x}_i
\end{cases}
\end{equation}

And an observer:

\begin{equation}
\hat{\bo{x}}_{i+1} = \bo{A}  \hat{\bo{x}}_i + \bo{B} \bo{u}_i + 
\bo{L} (\bo{y}_i - \bo{C} \hat{\bo{x}}_i)
\end{equation}

Remember that we can define observation error $\bo{e}_i = \hat{\bo{x}}_i - \bo{x}_i$ and write its dynamics:

\begin{equation}
\bo{e}_{i+1} = \bo{A} \bo{e}_i - 
\bo{L} \bo{C} \bo{e}_i
\end{equation}

Dual system (which is stable if and only if the original is stable), has form:

\begin{equation}
\varepsilon_{i+1} = \bo{A}^\top \varepsilon_i - 
\bo{C}^\top \bo{L}^\top \varepsilon_i
\end{equation}


\end{flushleft}
\end{frame}




\begin{frame}{Observability of Discrete LTI}
\framesubtitle{Dual system}
\begin{flushleft}

Dynamical system $\varepsilon_{i+1} = \bo{A}^\top \varepsilon_i - \bo{C}^\top \bo{L}^\top \varepsilon_i$, we can be represented as:

\begin{equation}
\begin{cases}
\varepsilon_{i+1} = \bo{A}^\top \varepsilon_i + \bo{C}^\top \bo{v}_i \\
\bo{v}_i = - \bo{L}^\top \varepsilon_i
\end{cases}
\end{equation}

Controllability matrix of this system is:

\begin{equation}
\mathcal{O}^\top = \begin{bmatrix}
    \bo{C}^\top &
    (\bo{A}^\top) \bo{C}^\top & ... &
    (\bo{A}^\top)^{n - 1} \bo{C}^\top
    \end{bmatrix}
\end{equation}

It is easier to represent this matrix in its transposed form:

\begin{equation}
\mathcal{O} = \begin{bmatrix}
    \bo{C} \\
    \bo{C}\bo{A}  \\ ... \\
    \bo{C}\bo{A}^{n - 1}
    \end{bmatrix}
\end{equation}

\end{flushleft}
\end{frame}


\begin{frame}{Observability criterion}
%\framesubtitle{Observability criterion}
\begin{flushleft}

\begin{block}{Observability}
For a system $\bo{x}_{i+1} = \bo{A}  \bo{x}_i + \bo{B} \bo{u}_i$ and $\bo{y}_i = \bo{C}  \bo{x}_i$, where $\bo{x} \in \R^n$, if the matrix $\mathcal{O} = \begin{bmatrix}
    \bo{C} \\
    \bo{C}\bo{A}  \\ ... \\
    \bo{C}\bo{A}^{n - 1}
    \end{bmatrix}$ is full column rank (i.e. $\text{rank}(\mathcal{O}) = n$), observation error can go to zero from any initial position, which means that \emph{the system is observable}.
\end{block}

\end{flushleft}
\end{frame}




\begin{frame}{Controllability, continuous-time (1)}
	% \framesubtitle{Limited control}
	\begin{flushleft}
		
		Let us consider matrix exponential $e^{\bo{A}t}$ is defined as a series:
		
		\begin{equation}
			e^{\bo{A}t} = \bo{I}+\bo{A}t+\frac{1}{2} \bo{A}\bo{A}t^2
			+\frac{1}{6} \bo{A}\bo{A}\bo{A}t^3 + ...
		\end{equation}
	
		Using Cayley–Hamilton we can observe that any powers of $\bo{A}$ higher than $n$ can be represented as a linear combination of lower powers. This gives us the following expression:
		%
		\begin{equation}
			e^{\bo{A}t} = \phi_0(t)\bo{I}+\phi_1(t)\bo{A}+\phi_2(t) \bo{A}^2 + ...
 + \phi_{n-1}(t) \bo{A}^{n-1}
		\end{equation}
		
		This allows us to re-write the forced state response:
		
		\begin{align*}
			\bo{x}(t) &= e^{\bo{A}t}  \bo{x}(0) + 
			\int_{0}^{t} e^{\bo{A}(t-\tau)} \bo{b}  u(\tau) d\tau
			\\
			\bo{x}(t) &= e^{\bo{A}t}  \bo{x}(0) + 
			\int_{0}^{t} 
			( \phi_0(t-\tau)\bo{I}+\phi_1(t-\tau)\bo{A}+ ... 
			\\
			&+ \phi_{n-1}(t-\tau) \bo{A}^{n-1} ) 
			\bo{b}  u(\tau) \ d\tau
		\end{align*}
		
		
	\end{flushleft}
\end{frame}



\begin{frame}{Controllability, continuous-time (2)}
	% \framesubtitle{Limited control}
	\begin{flushleft}
		
		\begin{align*}
			\bo{x}(t) &= e^{\bo{A}t}  \bo{x}(0) + 
			\int_{0}^{t} \phi_0(t-\tau)\bo{b} u(\tau) d\tau
			\\
			&+
			\int_{0}^{t} \phi_1(t-\tau)\bo{A}\bo{b}  u(\tau) d\tau+ ... 
		    \int_{0}^{t} \phi_{n-1}(t-\tau) \bo{A}^{n-1} 
			\bo{b}  u(\tau) d\tau 
		\end{align*}
	%
		\begin{align*}
		\bo{x}(t) -  e^{\bo{A}t}  \bo{x}(0) = 
		\begin{bmatrix}
			\bo{b} & \bo{A}\bo{b}& ... &\bo{A}^{n-1}\bo{b}
		\end{bmatrix}
		\begin{bmatrix}
			\int_{0}^{t} \phi_0(t-\tau) u(\tau) d\tau \\
			\int_{0}^{t} \phi_1(t-\tau) u(\tau) d\tau \\
			... \\
			\int_{0}^{t} \phi_{n-1}(t-\tau) u(\tau) d\tau \\
		\end{bmatrix}
		\end{align*}
		
		This shows that if controllability matrix is rank-deficient, it would not be possible to achieve some state from some initial condition.
		
	\end{flushleft}
\end{frame}




\begin{frame}{PBH controllability criterion}
	% \framesubtitle{Limited control}
	\begin{flushleft}
		
		There is an alternative way to test if pair $(\bo{A}, \ \bo{B})$ is controllable:
		
		\begin{block}{PBH controllability criterion}
			If for any $\lambda \in \mathbb{C}$, the the matrix $[ (\bo{A}-\lambda\bo{I}), \bo{B}]$ has full row rank, then the pair $(\bo{A}, \ \bo{B})$ is controllable.
		\end{block}
		
		\begin{itemize}
			\item If $\lambda$ is not an eigenvalue of $\bo{A}$, then $\text{det}(\bo{A}-\lambda\bo{I}) \neq 0$ and the matrix has full row rank.
			
			\item If $\text{det}(\bo{A}-\lambda\bo{I}) = 0$ and $\mathcal{V} = \text{null}(\bo{A}-\lambda\bo{I})$, $\text{dim}(\mathcal{V}) = 1$ and $\bo{v} \in \mathcal{V}$ , meaning $\lambda, \ \bo{v}$ are eigenvalue and eigenvector of $\bo{A}$, then in order for the criterion to hold the columns fo $\bo{B}$ should not all be orthogonal to $\bo{v}$: $\bo{v}\T \bo{B} \neq 0$.
			
			\item If eigenspace $\mathcal{V}$ is $k$-dimensional, the projection of $\bo{B}$ onto that eigenspace should also be $k$-dimensional.
			
		\end{itemize}
		
	\end{flushleft}
\end{frame}




\begin{frame}{Read more}
	
	\begin{itemize}
		
		\item Controllability and Observability (Rutgers University) \bref{https://www.ece.rutgers.edu/~gajic/psfiles/chap5.pdf}{https://www.ece.rutgers.edu/~gajic/psfiles/chap5.pdf}
		
	\end{itemize}
	
\end{frame}

\myqrframe


\begin{frame}
	
	\centering{\huge Appendix A: Analytical solution (recap)}
	
\end{frame}



\begin{frame}{Matrix exponential}
	% \framesubtitle{Limited control}
	\begin{flushleft}
		
		Exponential $e^a$ is defined as a series:
		
		\begin{equation}
			e^a =1+a+\frac{1}{2} a^2
			+\frac{1}{6} a^3 + ... = \sum_{n=0}^{\infty} \frac{1}{n!} a^n
		\end{equation}
		
		Matrix exponential $e^\bo{A}$ is defined as a series:
		
		\begin{equation}
			e^\bo{A} = \bo{I}+\bo{A}+\frac{1}{2} \bo{A}\bo{A}
			+\frac{1}{6} \bo{A}\bo{A}\bo{A} + ... = \sum_{n=0}^{\infty} \frac{1}{n!} \bo{A}^n
		\end{equation}
		
		
	\end{flushleft}
\end{frame}




\begin{frame}{Analytical solution to ODE}
	% \framesubtitle{Limited control}
	\begin{flushleft}
		
		An ODE of the form $\dot x = a x$ has analytical solution $x(t) = e^{at} x(0)$.
		
		\bigskip
		
		An ODE of the form $\dot{\bo{x}} = \bo{A}  \bo{x}$ has analytical solution $\bo{x}(t) = e^{\bo{A}t}  \bo{x}(0)$.
		
		\bigskip
		
		Let us check that this is a solution:
		
		\begin{align}
			\bo{x}(t) &= \left( \bo{I}+\bo{A}t+\frac{1}{2} \bo{A}\bo{A}t^2
			+\frac{1}{6} \bo{A}\bo{A}\bo{A}t^3 + ... \right)  \bo{x}(0) &
			\\
			\dot{\bo{x}}(t) &= \left( \bo{A}+\bo{A}\bo{A}t
			+\frac{1}{2}\bo{A}\bo{A}\bo{A}t^2 + ... \right)  \bo{x}(0) 	&		
			\\
			\dot{\bo{x}}(t) &= \bo{A} \left(\bo{I} +\bo{A}t
			+\frac{1}{2}\bo{A}\bo{A}t^2 + ... \right)  \bo{x}(0) 	&		
			\\
			\dot{\bo{x}}(t) &= \bo{A} e^{\bo{A}t}  \bo{x}(0) 	&		 			
			\\
			\dot{\bo{x}}(t) &= \bo{A} \bo{x}(t)  &\ \ \qed
		\end{align}
		
		
		
	\end{flushleft}
\end{frame}



\begin{frame}{Forced State Response (LTI) (1)}
	% \framesubtitle{Limited control}
	\begin{flushleft}
		
		An ODE of the form $\dot x = a x + bu(t)$ also has analytical solution. To find it, we first find the following derivative:
		%
		\begin{equation}
			\frac{d}{dt} \left( e^{-at} x(t) \right) = e^{-at} \dot x(t) - a e^{-at} x(t)
		\end{equation}
		
		Multiplying $\dot x = a x + bu(t)$ by $e^{-at}$ we see:
		
		\begin{align}
			e^{-at} \dot x = e^{-at} a x + e^{-at} bu(t)
			\\
			e^{-at} \dot x -  e^{-at} a x= e^{-at} bu(t)
			\\
			\frac{d}{dt} \left( e^{-at} x(t) \right) = e^{-at} bu(t)
			\\
			\int_{0}^{t} \frac{d}{d\tau} \left( e^{-a\tau} x(\tau) \right) d\tau = 	\int_{0}^{t}  e^{-a\tau} bu(\tau) d\tau
		\end{align}
		
		
		
	\end{flushleft}
\end{frame}



\begin{frame}{Forced State Response (LTI) (2)}
	% \framesubtitle{Limited control}
	\begin{flushleft}
		
		Continuing the derivation:
		
		\begin{align}
			\int_{0}^{t} \frac{d}{d\tau} \left( e^{-a\tau} x(\tau) \right) d\tau = 	\int_{0}^{t}  e^{-a\tau} bu(\tau) d\tau
			\\
			e^{-at} x(t) - x(0) = \int_{0}^{t}  e^{-a\tau} bu(\tau) d\tau
			\\
			x(t) = e^{at} x(0) + e^{at} \int_{0}^{t}  e^{-a\tau} bu(\tau) d\tau
			\\
			x(t) = e^{at} x(0) + \int_{0}^{t}  e^{a(t-\tau)} bu(\tau) d\tau
		\end{align}
		
		
		
	\end{flushleft}
\end{frame}




\begin{frame}{Forced State Response (LTI) (3)}
	% \framesubtitle{Limited control}
	\begin{flushleft}
		
		State-space equation $\dot{\bo{x}} = \bo{A}  \bo{x} + \bo{B}  \bo{u}(t)$ also has an analytical solution:
		
		\begin{equation}
			\bo{x}(t) = e^{\bo{A}t}  \bo{x}(0) + 
			\int_{0}^{t} e^{\bo{A}(t-\tau)} \bo{B}  \bo{u}(\tau) d\tau
		\end{equation}
		
		The same can be re-written as:
		
		\begin{equation}
			\bo{x}(t) = e^{\bo{A}t}  \bo{x}(0) + 
			e^{\bo{A}t} \int_{0}^{t} e^{-\bo{A}\tau} \bo{B}  \bo{u}(\tau) d\tau
		\end{equation}
		
	\end{flushleft}
\end{frame}





\end{document}
