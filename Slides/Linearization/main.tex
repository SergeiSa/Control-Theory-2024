\documentclass{beamer}

\pdfmapfile{+sansmathaccent.map}


\mode<presentation>
{
	\usetheme{Warsaw} % or try Darmstadt, Madrid, Warsaw, Rochester, CambridgeUS, ...
	\usecolortheme{seahorse} % or try seahorse, beaver, crane, wolverine, ...
	\usefonttheme{serif}  % or try serif, structurebold, ...
	\setbeamertemplate{navigation symbols}{}
	\setbeamertemplate{caption}[numbered]
} 


%%%%%%%%%%%%%%%%%%%%%%%%%%%%
% itemize settings


%%%%%%%%%%%%%%%%%%%%%%%%%%%%
% itemize settings

\definecolor{myhotpink}{RGB}{255, 80, 200}
\definecolor{mywarmpink}{RGB}{255, 60, 160}
\definecolor{mylightpink}{RGB}{255, 80, 200}
\definecolor{mypink}{RGB}{255, 30, 80}
\definecolor{mydarkpink}{RGB}{155, 25, 60}

\definecolor{mypaleblue}{RGB}{240, 240, 255}
\definecolor{mylightblue}{RGB}{120, 150, 255}
\definecolor{myblue}{RGB}{90, 90, 255}
\definecolor{mygblue}{RGB}{70, 110, 240}
\definecolor{mydarkblue}{RGB}{0, 0, 180}
\definecolor{myblackblue}{RGB}{40, 40, 120}

\definecolor{mygreen}{RGB}{0, 200, 0}
\definecolor{mydarkgreen}{RGB}{0, 120, 0}
\definecolor{mygreen2}{RGB}{245, 255, 230}

\definecolor{mygray}{gray}{0.8}
\definecolor{mydarkgray}{RGB}{80, 80, 160}

\definecolor{mydarkred}{RGB}{160, 30, 30}
\definecolor{mylightred}{RGB}{255, 150, 150}
\definecolor{myred}{RGB}{200, 110, 110}
\definecolor{myblackred}{RGB}{120, 40, 40}

\definecolor{mygreen}{RGB}{0, 200, 0}
\definecolor{mygreen2}{RGB}{205, 255, 200}

\definecolor{mydarkcolor}{RGB}{60, 25, 155}
\definecolor{mylightcolor}{RGB}{130, 180, 250}

\setbeamertemplate{itemize items}[default]

\setbeamertemplate{itemize item}{\color{myblackblue}$\blacksquare$}
\setbeamertemplate{itemize subitem}{\color{mygblue}$\blacktriangleright$}
\setbeamertemplate{itemize subsubitem}{\color{mygray}$\blacksquare$}

\setbeamercolor{palette quaternary}{fg=white,bg=mydarkgray}
\setbeamercolor{titlelike}{parent=palette quaternary}

\setbeamercolor{palette quaternary2}{fg=black,bg=mypaleblue}
\setbeamercolor{frametitle}{parent=palette quaternary2}

\setbeamerfont{frametitle}{size=\Large,series=\scshape}
\setbeamerfont{framesubtitle}{size=\normalsize,series=\upshape}





%%%%%%%%%%%%%%%%%%%%%%%%%%%%
% block settings

\setbeamercolor{block title}{bg=red!30,fg=black}

\setbeamercolor*{block title example}{bg=mygreen!40!white,fg=black}

\setbeamercolor*{block body example}{fg= black, bg= mygreen2}


%%%%%%%%%%%%%%%%%%%%%%%%%%%%
% URL settings
\hypersetup{
	colorlinks=true,
	linkcolor=blue,
	filecolor=blue,      
	urlcolor=blue,
}

%%%%%%%%%%%%%%%%%%%%%%%%%%

\renewcommand{\familydefault}{\rmdefault}

\usepackage{amsmath}
\usepackage{mathtools}

\usepackage{subcaption}

\usepackage{qrcode}

\DeclareMathOperator*{\argmin}{arg\,min}
\newcommand{\bo}[1] {\mathbf{#1}}

\newcommand{\R}{\mathbb{R}} 
\newcommand{\T}{^\top}     

\newcommand{\dx}[1] {\dot{\mathbf{#1}}}
\newcommand{\ma}[4] {\begin{bmatrix}
		#1 & #2 \\ #3 & #4
\end{bmatrix}}
\newcommand{\myvec}[2] {\begin{bmatrix}
		#1 \\ #2
\end{bmatrix}}
\newcommand{\myvecT}[2] {\begin{bmatrix}
		#1 & #2
\end{bmatrix}}


\newcommand{\mydate}{Spring 2023}

\newcommand{\mygit}{\textcolor{blue}{\href{https://github.com/SergeiSa/Control-Theory-Slides-Spring-2023}{github.com/SergeiSa/Control-Theory-Slides-Spring-2023}}}

\newcommand{\myqr}{ \textcolor{black}{\qrcode[height=1.5in]{https://github.com/SergeiSa/Control-Theory-Slides-Spring-2023}}
}

\newcommand{\myqrframe}{
	\begin{frame}
		\centerline{Lecture slides are available via Github, links are on Moodle}
		\bigskip
		\centerline{You can help improve these slides at:}
		\centerline{\mygit}
		\bigskip
		\myqr
	\end{frame}
}


\newcommand{\bref}[2] {\textcolor{blue}{\href{#1}{#2}}}

%%%%%%%%%%%%%%%%%%%%%%%%%%%%
% code settings

\usepackage{listings}
\usepackage{color}
% \definecolor{mygreen}{rgb}{0,0.6,0}
% \definecolor{mygray}{rgb}{0.5,0.5,0.5}
\definecolor{mymauve}{rgb}{0.58,0,0.82}
\lstset{ 
	backgroundcolor=\color{white},   % choose the background color; you must add \usepackage{color} or \usepackage{xcolor}; should come as last argument
	basicstyle=\footnotesize,        % the size of the fonts that are used for the code
	breakatwhitespace=false,         % sets if automatic breaks should only happen at whitespace
	breaklines=true,                 % sets automatic line breaking
	captionpos=b,                    % sets the caption-position to bottom
	commentstyle=\color{mygreen},    % comment style
	deletekeywords={...},            % if you want to delete keywords from the given language
	escapeinside={\%*}{*)},          % if you want to add LaTeX within your code
	extendedchars=true,              % lets you use non-ASCII characters; for 8-bits encodings only, does not work with UTF-8
	firstnumber=0000,                % start line enumeration with line 0000
	frame=single,	                   % adds a frame around the code
	keepspaces=true,                 % keeps spaces in text, useful for keeping indentation of code (possibly needs columns=flexible)
	keywordstyle=\color{blue},       % keyword style
	language=Octave,                 % the language of the code
	morekeywords={*,...},            % if you want to add more keywords to the set
	numbers=left,                    % where to put the line-numbers; possible values are (none, left, right)
	numbersep=5pt,                   % how far the line-numbers are from the code
	numberstyle=\tiny\color{mygray}, % the style that is used for the line-numbers
	rulecolor=\color{black},         % if not set, the frame-color may be changed on line-breaks within not-black text (e.g. comments (green here))
	showspaces=false,                % show spaces everywhere adding particular underscores; it overrides 'showstringspaces'
	showstringspaces=false,          % underline spaces within strings only
	showtabs=false,                  % show tabs within strings adding particular underscores
	stepnumber=2,                    % the step between two line-numbers. If it's 1, each line will be numbered
	stringstyle=\color{mymauve},     % string literal style
	tabsize=2,	                   % sets default tabsize to 2 spaces
	title=\lstname                   % show the filename of files included with \lstinputlisting; also try caption instead of title
}


%%%%%%%%%%%%%%%%%%%%%%%%%%%%
% URL settings
\hypersetup{
	colorlinks=false,
	linkcolor=blue,
	filecolor=blue,      
	urlcolor=blue,
}

%%%%%%%%%%%%%%%%%%%%%%%%%%

%%%%%%%%%%%%%%%%%%%%%%%%%%%%
% tikz settings

\usepackage{tikz}
\tikzset{every picture/.style={line width=0.75pt}}


\title{Linearization}
\subtitle{Control Theory, Lecture 11}
\author{by Sergei Savin}
\centering
\date{\mydate}



\begin{document}
\maketitle


\begin{frame}{Content}
\begin{itemize}
	\item Taylor expansion
	\item Linearization
	\item Linearization of Manipulator equations
\end{itemize}
\end{frame}






\begin{frame}{Taylor expansion around node, 1}
%	\framesubtitle{Linearization, Taylor Expansion}
	\begin{flushleft}
		
		Consider a non-linear dynamical system:
		%
		\begin{equation}
			\dot{\bo{x}} = \bo{f}(\bo{x}, \bo{u})
		\end{equation}
	
		If $\bo{x}_0$ and $\bo{u}_0$ represent a \emph{node}, i.e. $\bo{f}(\bo{x}_0, \bo{u}_0) = 0$, $\bo{x}_0 = \text{const}$, $\bo{u}_0 = \text{const}$, we can consider a Taylor expansion around that node:
	
		\begin{equation}
			\bo{f}(\bo{x}, \bo{u}) 	\sim 
			\frac{\partial \bo{f}}{\partial \bo{x}} (\bo{x} - \bo{x}_0) + 
			\frac{\partial \bo{f}}{\partial \bo{u}} (\bo{u} - \bo{u}_0) + \text{H.O.T.}
		\end{equation}
	
		Where $\bo{x}_0$ and $\bo{u}_0$ are expansion point. We define new variables $\bo{e}$ and $\bo{v}$ as distance from the expansion point:
		%
		\begin{align}
			\bo{e} = \bo{x} - \bo{x}_0, &\ \ \ \dot{\bo{e}} = \dot{\bo{x}}, \\
			\bo{v} = \bo{u} - \bo{u}_0. &\\
		\end{align}
		
		
	\end{flushleft}
\end{frame}


\begin{frame}{Taylor expansion around node, 2}
	%	\framesubtitle{Linearization, Taylor Expansion}
	\begin{flushleft}
		
		With that we can re-write the Taylor expansion:
		%
		\begin{align}
			\dot{\bo{e}} = \frac{\partial \bo{f}}{\partial \bo{x}} \bo{e} + 
			\frac{\partial \bo{f}}{\partial \bo{u}} \bo{v} + \text{H.O.T.}
		\end{align}		
	
		We can introduce notation:
		
		\begin{align}
			\bo{A} = \frac{\partial \bo{f}}{\partial \bo{x}},  \ \ \
			\bo{B} = \frac{\partial \bo{f}}{\partial \bo{u}}.
		\end{align}
		
		If we drop higher order terms from the Taylor expansion, we obtain \emph{linearization} of the system dynamics:
		
		\begin{align}
			\dot{\bo{e}} = \bo{A} \bo{e} + \bo{B} \bo{v}
		\end{align}	
	
		In this context, $\bo{x}_0$ and $\bo{u}_0$ is the \emph{linearization point}.
		
	\end{flushleft}
\end{frame}



\begin{frame}{Taylor expansion along a trajectory}
	%	\framesubtitle{Linearization, Taylor Expansion}
	\begin{flushleft}
		
		Consider a non-linear dynamical system:
		%
		\begin{equation}
			\dot{\bo{x}} = \bo{f}(\bo{x}, \bo{u})
		\end{equation}
		
		and a trajectory $\dot{\bo{x}}_0 = \bo{f}(\bo{x}_0, \bo{u}_0)$. We can consider a Taylor expansion along this trajectory:
		
		\begin{equation}
			\bo{f}(\bo{x}, \bo{u}) 	\sim \bo{f}(\bo{x}_0, \bo{u}_0) +
			\frac{\partial \bo{f}}{\partial \bo{x}} (\bo{x} - \bo{x}_0) + 
			\frac{\partial \bo{f}}{\partial \bo{u}} (\bo{u} - \bo{u}_0) + \text{H.O.T.}
		\end{equation}
		
		Since $\dot{\bo{e}} = \dot{\bo{x}} - \dot{\bo{x}}_0$, we re-write:
		%
		\begin{align}
			\dot{\bo{e}} \sim 
			\bo{A} \bo{e} + 
			\bo{B} \bo{v} + \text{H.O.T.}
		\end{align}		
		
		As before, we drop higher order terms and obtain linearization:
		
		\begin{align}
			\dot{\bo{e}} = 
			\bo{A} \bo{e} + 
			\bo{B} \bo{v}
		\end{align}		
	
	\end{flushleft}
\end{frame}




\begin{frame}{Affine expansion}
	%	\framesubtitle{Linearization, Taylor Expansion}
	\begin{flushleft}
		
		If we want to maintain our original variables, we can still use Taylor expansion:
		
		\begin{equation}
			\bo{f}(\bo{x}, \bo{u}) 	\sim \bo{f}(\bo{x}_0, \bo{u}_0) +
			\bo{A} (\bo{x} - \bo{x}_0) + 
			\bo{B} (\bo{u} - \bo{u}_0)
		\end{equation}
		
		Denoting $ \bo{f}(\bo{x}_0, \bo{u}_0) - \bo{A}\bo{x}_0 - \bo{B} \bo{u}_0 = \bo{c}$ and dropping H.O.T. we approximate the system as affine:
		
		\begin{equation}
			\dot{\bo{x}} = \bo{A} \bo{x} + 
			\bo{B} \bo{u} + \bo{c}
		\end{equation}
		
		
	\end{flushleft}
\end{frame}



\begin{frame}{Manipulator equation linearization, 1}
%\framesubtitle{Idea}
\begin{flushleft}

Consider Manipulator equation:

\begin{equation}
	\bo{H} \ddot{\bo{q}} + \bo{C} \dot{\bo{q}} + \bo{g} = \tau
\end{equation}

We will attempt to linearize it.

\bigskip

We begin by proposing the following new variables:

\begin{align}
	\bo{x} = 
	\begin{bmatrix}
		\bo{q} - \bo{q}_0 \\ 
		\dot{\bo{q}} - \dot{\bo{q}}_0
	\end{bmatrix},&  
\ \ \
	\bo{u} = \tau - \tau_0
	\\
	\bo{q} = \bo{S}_q \bo{x},&  
	\ \ \
	\dot{\bo{q}} = \bo{S}_v \bo{x}
\end{align}

where $\tau_0$ is chosen such that $\bo{C}(\dot{\bo{q}}_0, \bo{q}_0) \dot{\bo{q}}_0 + \bo{g}(\bo{q}_0) = \tau_0$, and $\bo{S}_q$ and $\bo{S}_v$ are choice matrices.


\end{flushleft}
\end{frame}



\begin{frame}{Manipulator equation linearization, 2}
%	\framesubtitle{Idea}
	\begin{flushleft}
		
		Next, we introduce function $\phi(\dot{\bo{q}}, \bo{q}, \tau) = \ddot{\bo{q}}$, expressed as:
		
		\begin{equation}
			\phi(\dot{\bo{q}}, \bo{q}, \tau) 
			=
			\bo{H}^{-1} (\tau - \bo{C} \dot{\bo{q}} - \bo{g} ) 
		\end{equation}
		
		Next, we write our dynamics as a first order ODE:
		
		\begin{equation}
			\frac{d}{dt} 
			\left(
			\begin{bmatrix}
				\bo{q} \\ \dot{\bo{q}} 
			\end{bmatrix}
			\right)
			=
			\begin{bmatrix}
				\dot{\bo{q}}  \\ 
				\phi(\dot{\bo{q}}, \bo{q}, \tau) 
			\end{bmatrix}
		\end{equation}
	
	\begin{equation}
	\dot{\bo{x}} 
	=
	\begin{bmatrix}
		\bo{S}_v \bo{x}  \\ 
		\phi(\bo{x}, \tau) 
	\end{bmatrix}
\end{equation}
		
		With that, we can find matrices $\bo{A}$ and $\bo{B}$.
		
	\end{flushleft}
\end{frame}



\begin{frame}{Manipulator equation linearization, 3}
%\framesubtitle{State matrix}
\begin{flushleft}

In this case, state matrices $\bo{A}$ and $\bo{B}$ become:


\begin{equation}
	\bo{A} = 
	\begin{bmatrix}
		\frac{\partial \dot{\bo{q}}}{\partial \bo{q}}  & 
		\frac{\partial \dot{\bo{q}}}{\partial \dot{\bo{q}}}
		 \\
		 \frac{\partial \phi}{\partial \bo{q}}  & 
		 \frac{\partial \phi}{\partial \dot{\bo{q}}}
	\end{bmatrix}
=
	\begin{bmatrix}
	0 & \bo{I}
	\\
	\frac{\partial \phi}{\partial \bo{q}}  & 
	\frac{\partial \phi}{\partial \dot{\bo{q}}}
\end{bmatrix}
\end{equation}


\begin{equation}
	\bo{B} = 
	\begin{bmatrix}
		\frac{\partial \dot{\bo{q}}}{\partial \tau} 
		\\
		\frac{\partial \phi}{\partial \tau} 
	\end{bmatrix}
	=
	\begin{bmatrix}
		0
		\\
		\bo{H}^{-1}
	\end{bmatrix}
\end{equation}

Thus. our task is to find the following jacobians: $\frac{\partial \phi}{\partial \bo{q}}$ and $\frac{\partial \phi}{\partial \dot{\bo{q}}}$.

\end{flushleft}
\end{frame}



\begin{frame}{Manipulator equation linearization, 4}
	%\framesubtitle{State matrix}
	\begin{flushleft}
		
		Let us find $\frac{\partial \phi}{\partial \bo{q}}$:
		
		\begin{align}
			\frac{\partial \phi}{\partial \bo{q}}
			&=
			\frac{\partial }{\partial \bo{q}} \left(\bo{H}^{-1} (\tau - \bo{C} \dot{\bo{q}} - \bo{g} ) \right)
			= \\
			&=
			\frac{\partial \bo{H}^{-1}}{\partial \bo{q}} (\tau - \bo{C} \dot{\bo{q}} - \bo{g} )
			+
			\bo{H}^{-1} \frac{\partial }{\partial \bo{q}} \left( \tau - \bo{C} \dot{\bo{q}} - \bo{g} \right)
		\end{align}
		
		If we evaluate $\frac{\partial \phi}{\partial \bo{q}}$ at the point $\bo{q} = \bo{q}_0$, $\dot{\bo{q}} = \dot{\bo{q}}_0$, $\tau = \tau_0$, we can use the fact that $\bo{C}(\dot{\bo{q}}_0, \bo{q}_0) \dot{\bo{q}}_0 + \bo{g}(\bo{q}_0) = \tau_0$ to avoid computing derivative $\frac{\partial \bo{H}^{-1}}{\partial \bo{q}}$:
		
		\begin{align}
			\frac{\partial \phi}{\partial \bo{q}}
			&=
			\bo{H}^{-1} \left( \tau - \frac{\partial \bo{C}\dot{\bo{q}}}{\partial \bo{q}}  - \frac{\partial \bo{g}}{\partial \bo{q}} \right)
		\end{align}
		
	\end{flushleft}
\end{frame}




\begin{frame}{Manipulator equation linearization, 5}
	%\framesubtitle{State matrix}
	\begin{flushleft}
		
		Let us find $\frac{\partial \phi}{\partial \dot{\bo{q}}}$:
		%
		\begin{align}
			\frac{\partial \phi}{\partial \dot{\bo{q}}}
			&=
			\frac{\partial }{\partial \dot{\bo{q}}} \left(\bo{H}^{-1} (\tau - \bo{C} \dot{\bo{q}} - \bo{g} ) \right)
			= \\
			&=
			-\bo{H}^{-1}\frac{\partial \bo{C} \dot{\bo{q}}}{\partial \dot{\bo{q}}}
		\end{align}
		
	With that, we expressed all jacobians. The rest is the same as in the general case we studied in the first slides.
		
	\end{flushleft}
\end{frame}



%\begin{equation}
%	\bo{A} = 
%	\begin{bmatrix}
%		0 & \bo{I} \\
%		\frac{\partial (\bo{H}^{-1} (\bo{T}\bo{u} -  \bo{C} \dot{\bo{q}} - \bo{g}) )}{\partial \bo{q}} 
%		&
%		\frac{\partial (\bo{H}^{-1} (\bo{T}\bo{u} -  \bo{C} \dot{\bo{q}} - \bo{g}) )}{\partial \dot{\bo{q}}}
%	\end{bmatrix}
%\end{equation}
%
%
%\begin{equation*}
%	\frac{\partial }{\partial \bo{q}} (\bo{H}^{-1} (\bo{T}\bo{u} -  \bo{C} \dot{\bo{q}} - \bo{g}) ) 
%	= 
%	\frac{\partial \bo{H}^{-1}}{\partial \bo{q}} (\bo{T}\bo{u} -  \bo{C} \dot{\bo{q}} - \bo{g})
%	+ 
%	\bo{H}^{-1} 
%	\frac{\partial }{\partial \bo{q}} 
%	(\bo{T}\bo{u} -  \bo{C} \dot{\bo{q}} - \bo{g}) )
%\end{equation*}
%
%\begin{equation}
%	\frac{\partial (\bo{H}^{-1} (\bo{T}\bo{u} -  \bo{C} \dot{\bo{q}} - \bo{g}) )}{\partial \dot{\bo{q}}} 
%	= 
%	-\bo{H}^{-1}\bo{C} - \bo{H}^{-1} 
%	\frac{\partial \bo{C}}{\partial \dot{\bo{q}}}  \dot{\bo{q}}
%\end{equation}
%
%\begin{equation}
%	\frac{\partial \bo{H}^{-1}}{\partial \bo{q}} = 
%	- \bo{H}^{-1} \frac{\partial \bo{H}}{\partial \bo{q}} 
%	\bo{H}^{-1}
%\end{equation}



%\begin{frame}{Linear control for nonlinear systems}
%	\framesubtitle{Linearization around node}
%	\begin{flushleft}
%		
%		In the general case it is:
%	
%	Let us make an assumption that our linearization point $\bo{q}_0$, $\dot{\bo{q}}_0$ and $\bo{u}_0$ is a node, meaning that $\ddot{\bo{q}}_0 = 0$, which implies:
%	
%	\begin{equation}
%		\bo{C} \dot{\bo{q}} + \bo{g} = \bo{T}\bo{u}
%	\end{equation}
%
%Then 
%	
%	\begin{equation*}
%		\frac{\partial }{\partial \bo{q}} (\bo{H}^{-1} (\bo{T}\bo{u} -  \bo{C} \dot{\bo{q}} - \bo{g}) ) 
%		= 
%		\frac{\partial \bo{H}^{-1}}{\partial \bo{q}} 0
%		+ 
%		\bo{H}^{-1} 
%		\frac{\partial }{\partial \bo{q}} 
%		(\bo{T}\bo{u} -  \bo{C} \dot{\bo{q}} - \bo{g}) )
%	\end{equation*}
%	
%	
%		
%	\end{flushleft}
%\end{frame}


%\begin{frame}{Linear control for nonlinear systems}
%\framesubtitle{Control matrix}
%\begin{flushleft}
%
%Control matrix $\bo{B}$ becomes:
%
%\begin{equation}
%    \bo{B} = 
%    \begin{bmatrix}
%    0\\
%    \frac{\partial (\bo{H}^{-1} (\bo{T}\bo{u} -  \bo{C} \dot{\bo{q}} - \bo{g}) )}{\partial \bo{u}} 
%    \end{bmatrix}
%    =
%        \begin{bmatrix}
%    0\\
%    \bo{H}^{-1} \bo{T}
%    \end{bmatrix}
%\end{equation}
%
%...and this does not look very clean and nice to use. Indeed, it is not easy or nice in practice.
%
%\end{flushleft}
%\end{frame}



\myqrframe

\end{document}
