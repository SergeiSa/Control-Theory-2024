\documentclass{beamer}

\pdfmapfile{+sansmathaccent.map}


\mode<presentation>
{
	\usetheme{Warsaw} % or try Darmstadt, Madrid, Warsaw, Rochester, CambridgeUS, ...
	\usecolortheme{seahorse} % or try seahorse, beaver, crane, wolverine, ...
	\usefonttheme{serif}  % or try serif, structurebold, ...
	\setbeamertemplate{navigation symbols}{}
	\setbeamertemplate{caption}[numbered]
} 


%%%%%%%%%%%%%%%%%%%%%%%%%%%%
% itemize settings


%%%%%%%%%%%%%%%%%%%%%%%%%%%%
% itemize settings

\definecolor{myhotpink}{RGB}{255, 80, 200}
\definecolor{mywarmpink}{RGB}{255, 60, 160}
\definecolor{mylightpink}{RGB}{255, 80, 200}
\definecolor{mypink}{RGB}{255, 30, 80}
\definecolor{mydarkpink}{RGB}{155, 25, 60}

\definecolor{mypaleblue}{RGB}{240, 240, 255}
\definecolor{mylightblue}{RGB}{120, 150, 255}
\definecolor{myblue}{RGB}{90, 90, 255}
\definecolor{mygblue}{RGB}{70, 110, 240}
\definecolor{mydarkblue}{RGB}{0, 0, 180}
\definecolor{myblackblue}{RGB}{40, 40, 120}

\definecolor{mygreen}{RGB}{0, 200, 0}
\definecolor{mydarkgreen}{RGB}{0, 120, 0}
\definecolor{mygreen2}{RGB}{245, 255, 230}

\definecolor{mygray}{gray}{0.8}
\definecolor{mydarkgray}{RGB}{80, 80, 160}

\definecolor{mydarkred}{RGB}{160, 30, 30}
\definecolor{mylightred}{RGB}{255, 150, 150}
\definecolor{myred}{RGB}{200, 110, 110}
\definecolor{myblackred}{RGB}{120, 40, 40}

\definecolor{mygreen}{RGB}{0, 200, 0}
\definecolor{mygreen2}{RGB}{205, 255, 200}

\definecolor{mydarkcolor}{RGB}{60, 25, 155}
\definecolor{mylightcolor}{RGB}{130, 180, 250}

\setbeamertemplate{itemize items}[default]

\setbeamertemplate{itemize item}{\color{myblackblue}$\blacksquare$}
\setbeamertemplate{itemize subitem}{\color{mygblue}$\blacktriangleright$}
\setbeamertemplate{itemize subsubitem}{\color{mygray}$\blacksquare$}

\setbeamercolor{palette quaternary}{fg=white,bg=mydarkgray}
\setbeamercolor{titlelike}{parent=palette quaternary}

\setbeamercolor{palette quaternary2}{fg=black,bg=mypaleblue}
\setbeamercolor{frametitle}{parent=palette quaternary2}

\setbeamerfont{frametitle}{size=\Large,series=\scshape}
\setbeamerfont{framesubtitle}{size=\normalsize,series=\upshape}





%%%%%%%%%%%%%%%%%%%%%%%%%%%%
% block settings

\setbeamercolor{block title}{bg=red!30,fg=black}

\setbeamercolor*{block title example}{bg=mygreen!40!white,fg=black}

\setbeamercolor*{block body example}{fg= black, bg= mygreen2}


%%%%%%%%%%%%%%%%%%%%%%%%%%%%
% URL settings
\hypersetup{
	colorlinks=true,
	linkcolor=blue,
	filecolor=blue,      
	urlcolor=blue,
}

%%%%%%%%%%%%%%%%%%%%%%%%%%

\renewcommand{\familydefault}{\rmdefault}

\usepackage{amsmath}
\usepackage{mathtools}

\usepackage{subcaption}

\usepackage{qrcode}

\DeclareMathOperator*{\argmin}{arg\,min}
\newcommand{\bo}[1] {\mathbf{#1}}

\newcommand{\R}{\mathbb{R}} 
\newcommand{\T}{^\top}     

\newcommand{\dx}[1] {\dot{\mathbf{#1}}}
\newcommand{\ma}[4] {\begin{bmatrix}
		#1 & #2 \\ #3 & #4
\end{bmatrix}}
\newcommand{\myvec}[2] {\begin{bmatrix}
		#1 \\ #2
\end{bmatrix}}
\newcommand{\myvecT}[2] {\begin{bmatrix}
		#1 & #2
\end{bmatrix}}


\newcommand{\mydate}{Spring 2023}

\newcommand{\mygit}{\textcolor{blue}{\href{https://github.com/SergeiSa/Control-Theory-Slides-Spring-2023}{github.com/SergeiSa/Control-Theory-Slides-Spring-2023}}}

\newcommand{\myqr}{ \textcolor{black}{\qrcode[height=1.5in]{https://github.com/SergeiSa/Control-Theory-Slides-Spring-2023}}
}

\newcommand{\myqrframe}{
	\begin{frame}
		\centerline{Lecture slides are available via Github, links are on Moodle}
		\bigskip
		\centerline{You can help improve these slides at:}
		\centerline{\mygit}
		\bigskip
		\myqr
	\end{frame}
}


\newcommand{\bref}[2] {\textcolor{blue}{\href{#1}{#2}}}

%%%%%%%%%%%%%%%%%%%%%%%%%%%%
% code settings

\usepackage{listings}
\usepackage{color}
% \definecolor{mygreen}{rgb}{0,0.6,0}
% \definecolor{mygray}{rgb}{0.5,0.5,0.5}
\definecolor{mymauve}{rgb}{0.58,0,0.82}
\lstset{ 
	backgroundcolor=\color{white},   % choose the background color; you must add \usepackage{color} or \usepackage{xcolor}; should come as last argument
	basicstyle=\footnotesize,        % the size of the fonts that are used for the code
	breakatwhitespace=false,         % sets if automatic breaks should only happen at whitespace
	breaklines=true,                 % sets automatic line breaking
	captionpos=b,                    % sets the caption-position to bottom
	commentstyle=\color{mygreen},    % comment style
	deletekeywords={...},            % if you want to delete keywords from the given language
	escapeinside={\%*}{*)},          % if you want to add LaTeX within your code
	extendedchars=true,              % lets you use non-ASCII characters; for 8-bits encodings only, does not work with UTF-8
	firstnumber=0000,                % start line enumeration with line 0000
	frame=single,	                   % adds a frame around the code
	keepspaces=true,                 % keeps spaces in text, useful for keeping indentation of code (possibly needs columns=flexible)
	keywordstyle=\color{blue},       % keyword style
	language=Octave,                 % the language of the code
	morekeywords={*,...},            % if you want to add more keywords to the set
	numbers=left,                    % where to put the line-numbers; possible values are (none, left, right)
	numbersep=5pt,                   % how far the line-numbers are from the code
	numberstyle=\tiny\color{mygray}, % the style that is used for the line-numbers
	rulecolor=\color{black},         % if not set, the frame-color may be changed on line-breaks within not-black text (e.g. comments (green here))
	showspaces=false,                % show spaces everywhere adding particular underscores; it overrides 'showstringspaces'
	showstringspaces=false,          % underline spaces within strings only
	showtabs=false,                  % show tabs within strings adding particular underscores
	stepnumber=2,                    % the step between two line-numbers. If it's 1, each line will be numbered
	stringstyle=\color{mymauve},     % string literal style
	tabsize=2,	                   % sets default tabsize to 2 spaces
	title=\lstname                   % show the filename of files included with \lstinputlisting; also try caption instead of title
}


%%%%%%%%%%%%%%%%%%%%%%%%%%%%
% URL settings
\hypersetup{
	colorlinks=false,
	linkcolor=blue,
	filecolor=blue,      
	urlcolor=blue,
}

%%%%%%%%%%%%%%%%%%%%%%%%%%

%%%%%%%%%%%%%%%%%%%%%%%%%%%%
% tikz settings

\usepackage{tikz}
\tikzset{every picture/.style={line width=0.75pt}}


\title{Discrete Dynamics}
\subtitle{Control Theory, Lecture 6}
\author{by Sergei Savin}
\centering
\date{\mydate}



\begin{document}
\maketitle


\begin{frame}{Content}

\begin{itemize}
\item Discrete Dynamics
\item Stability of the Discrete Dynamics
\item CT-LTI: Analytical solution
\begin{itemize}
    \item Matrix exponential
    \item Analytical solution
    \item Forced state Response
\end{itemize}
\item Discretization
\begin{itemize}
	\item Exact
	\item Zero order hold
\end{itemize}
\item Read more
\end{itemize}

\end{frame}





\begin{frame}{Discrete Dynamics}
% \framesubtitle{Part 1}
\begin{flushleft}

The following dynamical system is called \emph{discrete}:

\begin{equation}
    \bo{x}_{i+1} = \bo{A}\bo{x}_i + \bo{B}\bo{u}_i
\end{equation}

Note that those:

\begin{itemize}
    \item have no derivatives in the equation;
    \item are easily simulated.
\end{itemize}

\bigskip

The affine control for this system can be given as:

\begin{equation}
    \bo{u}_i = -\bo{K}\bo{x}_i + \bo{u}_i^*
\end{equation}

\end{flushleft}
\end{frame}




\begin{frame}{Stability of the Discrete Dynamics}
	\framesubtitle{Real eigenvalues}
	\begin{flushleft}
		
		Let us consider stability of the discrete dynamical system where matrix $\bo{A}$ has purely real eigenvalues:
		
		\begin{equation}
			\bo{x}_{i+1} = \bo{A}\bo{x}_i
		\end{equation}		
		
		With eigendecomposition $\bo{A} = \bo{V}^{-1} \bo{D} \bo{V}$ (where $\bo{D}$ is a diagonal matrix with eigenvalues $\lambda_j$ of $\bo{A}$ on its diagonal)  and introducing notation $\bo{z}_i = \bo{V}\bo{x}_i$ we get:
		
		\begin{equation}
			\bo{x}_{i+1} = \bo{V}^{-1} \bo{D} \bo{V}\bo{x}_i
		\end{equation}
		\begin{equation}
			\bo{z}_{i+1} =  \bo{D} \bo{z}_i
		\end{equation}
		
		Meaning that the dynamics became a system of independent scalar equations $z_{j, i+1} =  \lambda_j z_{j, i}$. 
		
		
	\end{flushleft}
\end{frame}




\begin{frame}{Stability of the Discrete Dynamics}
	\framesubtitle{Real eigenvalues}
	\begin{flushleft}
		
		Thus, with $z_{j, i+1} =  \lambda_j z_{j, i}$ we can find now the absolute value of the scalars $z_{j}$ will dwindle with time iff $|\lambda_j| < 1$:
		
		\begin{equation}
			\left| \frac{z_{j, i+1}}{z_{j, i}} \right | =   | \lambda_j |
		\end{equation}
		 
		
	\end{flushleft}
\end{frame}



\begin{frame}{Stability of the Discrete Dynamics}
	\framesubtitle{2x2 system}
	\begin{flushleft}
		
		Let us consider stability of the discrete dynamical system with a 2-by-2 matrix $\bo{A}$:
		
		\begin{equation}
			\begin{bmatrix}
				x_{1, i+1} \\ x_{2, i+1}
			\end{bmatrix}
			= 
			\begin{bmatrix}
				\alpha & -\beta \\ \beta & \alpha
			\end{bmatrix}     
			\begin{bmatrix}
				x_{1, i} \\ x_{2, i}
			\end{bmatrix}
		\end{equation}	
		
		Let us find norms of $\begin{bmatrix}
			x_{1, i+1} \\ x_{2, i+1}
		\end{bmatrix}$ and $\begin{bmatrix}
		x_{1, i} \\ x_{2, i}
	\end{bmatrix}$:


		\begin{equation}
		\left | \left|
		\begin{bmatrix}
			x_{1, i} \\ x_{2, i}
		\end{bmatrix}
		\right | \right|^2
			= 
	 	x_{1, i}^2 + x_{2, i}^2
		\end{equation}

		\begin{equation}
			\left | \left|
			\begin{bmatrix}
				x_{1, i+1} \\ x_{2, i+1}
			\end{bmatrix}
			\right | \right|^2
			= 
			(\alpha^2 + \beta^2) (x_{1, i}^2 + x_{2, i}^2)
		\end{equation}
		
		
	\end{flushleft}
\end{frame}



\begin{frame}{Stability of the Discrete Dynamics}
	\framesubtitle{2x2 system}
	\begin{flushleft}
		
		We can find the ratio of the norms of $\begin{bmatrix}
			x_{1, i+1} \\ x_{2, i+1}
		\end{bmatrix}$ and $\begin{bmatrix}
			x_{1, i} \\ x_{2, i}
		\end{bmatrix}$: 
		
		
		
		\begin{equation}
			\left | \left|
			\begin{bmatrix}
				x_{1, i+1} \\ x_{2, i+1}
			\end{bmatrix}
			\right | \right|^2 
			/ 
			\left | \left|
			\begin{bmatrix}
				x_{1, i} \\ x_{2, i}
			\end{bmatrix}
			\right | \right|^2
			= 
			\alpha^2 + \beta^2
		\end{equation}
		
		Remembering that eigenvalues of the system are $\lambda  = \alpha \pm j\beta$, we can rewrite the expression above as:
		
		\begin{equation}
	\left | \left|
	\begin{bmatrix}
		x_{1, i+1} \\ x_{2, i+1}
	\end{bmatrix}
	\right | \right|^2 
	/ 
	\left | \left|
	\begin{bmatrix}
		x_{1, i} \\ x_{2, i}
	\end{bmatrix}
	\right | \right|^2
	= 
	| \lambda |^2
		\end{equation}		
		
		We can see that the norm of the variable $\bo{x}$ will dwindle with time iff $|\lambda| < 1$.
		
	\end{flushleft}
\end{frame}




\begin{frame}{Stability of the Discrete Dynamics}
\begin{flushleft}

General stability criterion is given below:

\bigskip

\begin{block}{Stability criterion}
In general, discrete system $\bo{x}_{i+1} = \bo{A}\bo{x}_i$ is stable as long as the eigenvalues of $\bo{A}$ are smaller than 1 by absolute value: $|\lambda_i(\bo{A})| \leq 1, \; \forall i$.
\end{block}

\end{flushleft}
\end{frame}



\begin{frame}
	
	\centering{\huge CT-LTI: Analytical solution}
	
\end{frame}





\begin{frame}{Matrix exponential}
	% \framesubtitle{Limited control}
	\begin{flushleft}
		
		Exponential $e^a$ is defined as a series:
		
		\begin{equation}
			e^a =1+a+\frac{1}{2} a^2
			+\frac{1}{6} a^3 + ... = \sum_{n=0}^{\infty} \frac{1}{n!} a^n
		\end{equation}
		
		Matrix exponential $e^\bo{A}$ is defined as a series:
		
		\begin{equation}
			e^\bo{A} = \bo{I}+\bo{A}+\frac{1}{2} \bo{A}\bo{A}
			+\frac{1}{6} \bo{A}\bo{A}\bo{A} + ... = \sum_{n=0}^{\infty} \frac{1}{n!} \bo{A}^n
		\end{equation}
		
		
	\end{flushleft}
\end{frame}




\begin{frame}{Analytical solution to ODE}
	% \framesubtitle{Limited control}
	\begin{flushleft}
		
		An ODE of the form $\dot x = a x$ has analytical solution $x(t) = e^{at} x(0)$.
		
		\bigskip
		
		An ODE of the form $\dot{\bo{x}} = \bo{A}  \bo{x}$ has analytical solution $\bo{x}(t) = e^{\bo{A}t}  \bo{x}(0)$.
		
		\bigskip
		
		Let us check that this is a solution:
		
		\begin{align}
			\bo{x}(t) &= \left( \bo{I}+\bo{A}t+\frac{1}{2} \bo{A}\bo{A}t^2
			+\frac{1}{6} \bo{A}\bo{A}\bo{A}t^3 + ... \right)  \bo{x}(0) &
			\\
			\dot{\bo{x}}(t) &= \left( \bo{A}+\bo{A}\bo{A}t
			+\frac{1}{2}\bo{A}\bo{A}\bo{A}t^2 + ... \right)  \bo{x}(0) 	&		
			\\
			\dot{\bo{x}}(t) &= \bo{A} \left(\bo{I} +\bo{A}t
			+\frac{1}{2}\bo{A}\bo{A}t^2 + ... \right)  \bo{x}(0) 	&		
			\\
			\dot{\bo{x}}(t) &= \bo{A} e^{\bo{A}t}  \bo{x}(0) 	&		 			
			\\
			\dot{\bo{x}}(t) &= \bo{A} \bo{x}(t)  &\ \ \qed
		\end{align}
		
		
		
	\end{flushleft}
\end{frame}



\begin{frame}{Forced State Response (LTI) (1)}
	% \framesubtitle{Limited control}
	\begin{flushleft}
		
		An ODE of the form $\dot x = a x + bu(t)$ also has analytical solution. To find it, we first find the following derivative:
		%
		\begin{equation}
			\frac{d}{dt} \left( e^{-at} x(t) \right) = e^{-at} \dot x(t) - a e^{-at} x(t)
		\end{equation}
		
		Multiplying $\dot x = a x + bu(t)$ by $e^{-at}$ we see:
		
		\begin{align}
			e^{-at} \dot x = e^{-at} a x + e^{-at} bu(t)
			\\
			e^{-at} \dot x -  e^{-at} a x= e^{-at} bu(t)
			\\
			\frac{d}{dt} \left( e^{-at} x(t) \right) = e^{-at} bu(t)
			\\
			\int_{0}^{t} \frac{d}{d\tau} \left( e^{-a\tau} x(\tau) \right) d\tau = 	\int_{0}^{t}  e^{-a\tau} bu(\tau) d\tau
		\end{align}
		
		
		
	\end{flushleft}
\end{frame}



\begin{frame}{Forced State Response (LTI) (2)}
	% \framesubtitle{Limited control}
	\begin{flushleft}
		
		Continuing the derivation:
		
		\begin{align}
			\int_{0}^{t} \frac{d}{d\tau} \left( e^{-a\tau} x(\tau) \right) d\tau = 	\int_{0}^{t}  e^{-a\tau} bu(\tau) d\tau
			\\
			e^{-at} x(t) - x(0) = \int_{0}^{t}  e^{-a\tau} bu(\tau) d\tau
			\\
			x(t) = e^{at} x(0) + e^{at} \int_{0}^{t}  e^{-a\tau} bu(\tau) d\tau
			\\
			x(t) = e^{at} x(0) + \int_{0}^{t}  e^{a(t-\tau)} bu(\tau) d\tau
		\end{align}
		
		
		
	\end{flushleft}
\end{frame}




\begin{frame}{Forced State Response (LTI) (3)}
	% \framesubtitle{Limited control}
	\begin{flushleft}
		
		State-space equation $\dot{\bo{x}} = \bo{A}  \bo{x} + \bo{B}  \bo{u}(t)$ also has an analytical solution:
		
		\begin{equation}
			\bo{x}(t) = e^{\bo{A}t}  \bo{x}(0) + 
			\int_{0}^{t} e^{\bo{A}(t-\tau)} \bo{B}  \bo{u}(\tau) d\tau
		\end{equation}
		
		The same can be re-written as:
		
		\begin{equation}
			\bo{x}(t) = e^{\bo{A}t}  \bo{x}(0) + 
			e^{\bo{A}t} \int_{0}^{t} e^{-\bo{A}\tau} \bo{B}  \bo{u}(\tau) d\tau
		\end{equation}
		
	\end{flushleft}
\end{frame}






\begin{frame}
	
		\centering{\huge Discretization}
	
\end{frame}





\begin{frame}{From analytical solution to discrete dynamics}
	% \framesubtitle{Limited control}
	\begin{flushleft}
		Given a solution to a state-space system we can consider how the system evolves from the point $t=0$ to the point $t=\Delta t$, assuming that $\bo{u}(t) = \bo{u}_0 = \text{const}, \ t \in [ 0, \ \Delta t ]$, and denoting $\bo{x}(0) = \bo{x}_0$ and $\bo{x}(\Delta t) = \bo{x}_1$:
		
		\begin{equation}
			\bo{x}_1 = e^{\bo{A}\Delta t}  \bo{x}_0 + 
			e^{\bo{A} \Delta t} \int_{0}^{\Delta t} e^{-\bo{A}\tau} \bo{B}  \bo{u}_0 d\tau
		\end{equation}
		
		Now we denote:
		%
		\begin{align}
			\bar{\bo{A}} = e^{\bo{A}\Delta t}, \ \ \
			\bar{\bo{B}} = e^{\bo{A} \Delta t} \int_{0}^{\Delta t} e^{-\bo{A}\tau} \bo{B} d\tau
		\end{align}
	
		We get:
		%
		\begin{equation}
			\bo{x}_1 = \bar{\bo{A}}  \bo{x}_0 + \bar{\bo{B}}  \bo{u}_0
		\end{equation}		
		
	\end{flushleft}
\end{frame}




\begin{frame}{Discretization via finite differences}
%\framesubtitle{Finite difference}
\begin{flushleft}

Consider linear time-invariant autonomous system:

\begin{equation}
    \dot {\bo{x}} = \bo{A} \bo{x}
\end{equation}


The time derivative $\dot {\bo{x}}$ can be replaces with a finite difference:

\begin{equation}
\dot {\bo{x}} \approx \frac{1}{\Delta t}(\bo{x}(t + \Delta t) - \bo{x}(t))
\end{equation}

Note that we could have also used other definitions of a finite difference:

\begin{equation}
\dot {\bo{x}} \approx \frac{1}{\Delta t}(\bo{x}(t + 0.5\Delta t) - \bo{x}(t - 0.5\Delta t))
\end{equation}

or

\begin{equation}
\dot {\bo{x}} \approx \frac{1}{\Delta t}(\bo{x}(t) - \bo{x}(t - \Delta t))
\end{equation}

\end{flushleft}
\end{frame}




\begin{frame}{Zero-order hold (1)}
%\framesubtitle{Finite difference notation}
\begin{flushleft}
We can introduce notation:

\begin{equation}
\begin{cases}
\bo{x}_0 = \bo{x}(0) \\
\bo{x}_1 = \bo{x}(\Delta t) \\
\bo{x}_2 = \bo{x}(2\Delta t) \\
... \\
\bo{x}_n = \bo{x}(n\Delta t) 
\end{cases}
\end{equation}

We say that $\bo{x}_i$ is the value of $\bo{x}$ at the time step $i$. Then the finite difference can be written, for example, as follows:

\begin{equation}
\dot {\bo{x}} \approx \frac{1}{\Delta t}(\bo{x}_{i+1} - \bo{x}_i)
\end{equation}

\end{flushleft}
\end{frame}



\begin{frame}{Zero-order hold (2)}
%\framesubtitle{Finite difference in an autonomous LTI}
\begin{flushleft}

We can rewrite our original autonomous LTI as follows:

\begin{equation}
\frac{1}{\Delta t}(\bo{x}_{i + 1} - \bo{x}_i) = \bo{A} \bo{x}_i
\end{equation}
Isolating $\bo{x}_{i + 1}$ on the left hand side, we get:
\begin{equation}
\bo{x}_{i + 1} = (\bo{A} \Delta t + \bo{I}) \bo{x}_i
\end{equation}

\noindent\rule{11cm}{0.4pt}

Or alternatively:

\begin{equation}
\frac{1}{\Delta t}(\bo{x}_{i + 1} - \bo{x}_i) = \bo{A} \bo{x}_{i + 1}
\end{equation}
Isolating $\bo{x}_{i + 1}$ on the left hand side, we get:
\begin{equation}
\bo{x}_{i + 1} = (\bo{I} - \bo{A} \Delta t)^{-1} \bo{x}_i 
\end{equation}

\end{flushleft}
\end{frame}




\begin{frame}{Zero-order hold (3)}
%\framesubtitle{Zero order hold}
\begin{flushleft}

Defining \emph{discrete state space matrix} $\bar{\bo{A}}$ and \emph{discrete control matrix} $\bar{\bo{B}}$ as follows:

\begin{equation}
\bar{\bo{A}} = \bo{A} \Delta t + \bo{I}
\end{equation}
\begin{equation}
\bar{\bo{B}} = \bo{B} \Delta t
\end{equation}
%
We get discrete dynamics:

\begin{equation}
\bo{x}_{i+1} = \bar{\bo{A}} \bo{x}_i + \bar{\bo{B}} \mathbf u_i
\end{equation}

This way of defining discrete dynamics is called \emph{zero order hold (ZOH)}.

\end{flushleft}
\end{frame}


\begin{frame}{Zero-order hold (4)}
%\framesubtitle{Zero order hold vs First order hold}
\begin{flushleft}

Graphically, we can understand what zero order hold is, by comparing it to the first order hold:

\begin{figure} [h!]
\begin{center}
\includegraphics[width=3.5in]{ZOH.PNG}
\end{center} 
\caption{Different types of discretization} \label{F:ZOH}
\end{figure}

\end{flushleft}
\end{frame}


%\begin{frame}{ZOH and other types of discretization}
%\framesubtitle{Exact discretization}
%\begin{flushleft}
%
%Let the discrete state $\bo{x}_i$ correspond to continuous state $\bo{x}$ at the moment of time $t_i$. Then, we can say that the discretization is \emph{exact} the following holds for any solution $\bo{x}(t)$
%
%\begin{equation}
%\bo{x}_0 = \bo{x}(t_0) \rightarrow 
%\bo{x}_i = \bo{x}(t_i), \ \forall i
%\end{equation}
%
%We can compute the exact discretization as follows:
%
%\begin{equation}
%\bar{\bo{A}} = e^{\bo{A} \Delta t}
%\end{equation}
%\begin{equation}
%\bar{\bo{B}} = \bo{B} \int_{t_0}^{t_0 + \Delta t} e^{\bo{A} s} ds
%\end{equation}
%
%
%\end{flushleft}
%\end{frame}









\begin{frame}{Read more}

\begin{itemize}
\item  \bref{http://cse.lab.imtlucca.it/~bemporad/teaching/ac/pdf/04a-TD_sys.pdf}{Automatic Control 1 Discrete-time linear systems}, Prof. Alberto Bemporad, University of Trento

\item \bref{https://web.mit.edu/2.14/www/Handouts/StateSpaceResponse.pdf}{MIT 2.14, State Space Response}

\end{itemize}

\end{frame}



\myqrframe

\end{document}
