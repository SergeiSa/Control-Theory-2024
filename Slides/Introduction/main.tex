\documentclass{beamer}

\pdfmapfile{+sansmathaccent.map}


\mode<presentation>
{
	\usetheme{Warsaw} % or try Darmstadt, Madrid, Warsaw, Rochester, CambridgeUS, ...
	\usecolortheme{seahorse} % or try seahorse, beaver, crane, wolverine, ...
	\usefonttheme{serif}  % or try serif, structurebold, ...
	\setbeamertemplate{navigation symbols}{}
	\setbeamertemplate{caption}[numbered]
} 


%%%%%%%%%%%%%%%%%%%%%%%%%%%%
% itemize settings


%%%%%%%%%%%%%%%%%%%%%%%%%%%%
% itemize settings

\definecolor{myhotpink}{RGB}{255, 80, 200}
\definecolor{mywarmpink}{RGB}{255, 60, 160}
\definecolor{mylightpink}{RGB}{255, 80, 200}
\definecolor{mypink}{RGB}{255, 30, 80}
\definecolor{mydarkpink}{RGB}{155, 25, 60}

\definecolor{mypaleblue}{RGB}{240, 240, 255}
\definecolor{mylightblue}{RGB}{120, 150, 255}
\definecolor{myblue}{RGB}{90, 90, 255}
\definecolor{mygblue}{RGB}{70, 110, 240}
\definecolor{mydarkblue}{RGB}{0, 0, 180}
\definecolor{myblackblue}{RGB}{40, 40, 120}

\definecolor{mygreen}{RGB}{0, 200, 0}
\definecolor{mygreen2}{RGB}{245, 255, 230}

\definecolor{mygray}{gray}{0.8}
\definecolor{mydarkgray}{RGB}{80, 80, 160}

\definecolor{mydarkred}{RGB}{160, 30, 30}
\definecolor{mylightred}{RGB}{255, 150, 150}
\definecolor{myred}{RGB}{200, 110, 110}
\definecolor{myblackred}{RGB}{120, 40, 40}

\definecolor{mygreen}{RGB}{0, 200, 0}
\definecolor{mygreen2}{RGB}{205, 255, 200}

\definecolor{mydarkcolor}{RGB}{60, 25, 155}
\definecolor{mylightcolor}{RGB}{130, 180, 250}

\setbeamertemplate{itemize items}[default]

\setbeamertemplate{itemize item}{\color{myblackblue}$\blacksquare$}
\setbeamertemplate{itemize subitem}{\color{mygblue}$\blacktriangleright$}
\setbeamertemplate{itemize subsubitem}{\color{mygray}$\blacksquare$}

\setbeamercolor{palette quaternary}{fg=white,bg=mydarkgray}
\setbeamercolor{titlelike}{parent=palette quaternary}

\setbeamercolor{palette quaternary2}{fg=black,bg=mypaleblue}
\setbeamercolor{frametitle}{parent=palette quaternary2}

\setbeamerfont{frametitle}{size=\Large,series=\scshape}
\setbeamerfont{framesubtitle}{size=\normalsize,series=\upshape}





%%%%%%%%%%%%%%%%%%%%%%%%%%%%
% block settings

\setbeamercolor{block title}{bg=red!30,fg=black}

\setbeamercolor*{block title example}{bg=mygreen!40!white,fg=black}

\setbeamercolor*{block body example}{fg= black, bg= mygreen2}


%%%%%%%%%%%%%%%%%%%%%%%%%%%%
% URL settings
\hypersetup{
	colorlinks=true,
	linkcolor=blue,
	filecolor=blue,      
	urlcolor=blue,
}

%%%%%%%%%%%%%%%%%%%%%%%%%%

\renewcommand{\familydefault}{\rmdefault}

\usepackage{amsmath}
\usepackage{mathtools}

\usepackage{subcaption}

\usepackage{qrcode}

\DeclareMathOperator*{\argmin}{arg\,min}
\newcommand{\bo}[1] {\mathbf{#1}}

\newcommand{\R}{\mathbb{R}} 
\newcommand{\T}{^\top}     

\newcommand{\dx}[1] {\dot{\mathbf{#1}}}
\newcommand{\ma}[4] {\begin{bmatrix}
		#1 & #2 \\ #3 & #4
\end{bmatrix}}
\newcommand{\myvec}[2] {\begin{bmatrix}
		#1 \\ #2
\end{bmatrix}}
\newcommand{\myvecT}[2] {\begin{bmatrix}
		#1 & #2
\end{bmatrix}}


\newcommand{\mydate}{Spring 2023}

\newcommand{\mygit}{\textcolor{blue}{\href{https://github.com/SergeiSa/Control-Theory-Slides-Spring-2023}{github.com/SergeiSa/Control-Theory-Slides-Spring-2023}}}

\newcommand{\myqr}{ \textcolor{black}{\qrcode[height=1.5in]{https://github.com/SergeiSa/Control-Theory-Slides-Spring-2023}}
}

\newcommand{\myqrframe}{
	\begin{frame}
		\centerline{Lecture slides are available via Github, links are on Moodle}
		\bigskip
		\centerline{You can help improve these slides at:}
		\centerline{\mygit}
		\bigskip
		\myqr
	\end{frame}
}


\newcommand{\bref}[2] {\textcolor{blue}{\href{#1}{#2}}}

%%%%%%%%%%%%%%%%%%%%%%%%%%%%
% code settings

\usepackage{listings}
\usepackage{color}
% \definecolor{mygreen}{rgb}{0,0.6,0}
% \definecolor{mygray}{rgb}{0.5,0.5,0.5}
\definecolor{mymauve}{rgb}{0.58,0,0.82}
\lstset{ 
	backgroundcolor=\color{white},   % choose the background color; you must add \usepackage{color} or \usepackage{xcolor}; should come as last argument
	basicstyle=\footnotesize,        % the size of the fonts that are used for the code
	breakatwhitespace=false,         % sets if automatic breaks should only happen at whitespace
	breaklines=true,                 % sets automatic line breaking
	captionpos=b,                    % sets the caption-position to bottom
	commentstyle=\color{mygreen},    % comment style
	deletekeywords={...},            % if you want to delete keywords from the given language
	escapeinside={\%*}{*)},          % if you want to add LaTeX within your code
	extendedchars=true,              % lets you use non-ASCII characters; for 8-bits encodings only, does not work with UTF-8
	firstnumber=0000,                % start line enumeration with line 0000
	frame=single,	                   % adds a frame around the code
	keepspaces=true,                 % keeps spaces in text, useful for keeping indentation of code (possibly needs columns=flexible)
	keywordstyle=\color{blue},       % keyword style
	language=Octave,                 % the language of the code
	morekeywords={*,...},            % if you want to add more keywords to the set
	numbers=left,                    % where to put the line-numbers; possible values are (none, left, right)
	numbersep=5pt,                   % how far the line-numbers are from the code
	numberstyle=\tiny\color{mygray}, % the style that is used for the line-numbers
	rulecolor=\color{black},         % if not set, the frame-color may be changed on line-breaks within not-black text (e.g. comments (green here))
	showspaces=false,                % show spaces everywhere adding particular underscores; it overrides 'showstringspaces'
	showstringspaces=false,          % underline spaces within strings only
	showtabs=false,                  % show tabs within strings adding particular underscores
	stepnumber=2,                    % the step between two line-numbers. If it's 1, each line will be numbered
	stringstyle=\color{mymauve},     % string literal style
	tabsize=2,	                   % sets default tabsize to 2 spaces
	title=\lstname                   % show the filename of files included with \lstinputlisting; also try caption instead of title
}


%%%%%%%%%%%%%%%%%%%%%%%%%%%%
% URL settings
\hypersetup{
	colorlinks=false,
	linkcolor=blue,
	filecolor=blue,      
	urlcolor=blue,
}

%%%%%%%%%%%%%%%%%%%%%%%%%%

%%%%%%%%%%%%%%%%%%%%%%%%%%%%
% tikz settings

\usepackage{tikz}
\tikzset{every picture/.style={line width=0.75pt}}


\title{ODE and State Space}
\subtitle{Control Theory, Lecture 1}
\author{by Sergei Savin}
\centering
\date{\mydate}



\begin{document}
\maketitle


%\begin{frame}{Content}
%
%\begin{itemize}
%\item Motivation
%\item Ordinary differential equations
%    \begin{itemize}
%    \item 1st order
%    \item n-th order
%    \end{itemize}
%\item Linear differential equations
%    \begin{itemize}
%    \item 1st order
%    \item n-th order
%    \end{itemize}
%\item Changing n-th order ODE to a State-Space form
%\item State-Space to ODE
%\item Read more
%\end{itemize}
%
%\end{frame}




\begin{frame}{Ordinary differential equations, 1st order}
%\framesubtitle{}
\begin{flushleft}

Let us remember the normal form of first-order \emph{ordinary differential equations (ODEs):}

\begin{equation}
    \dot{\bo{x}} = \bo{f} (\bo{x}, t)
\end{equation}

where $\bo{x} = \bo{x}(t)$ is the solution of the equation and $t$ is a free variable (usually - time).

\bigskip

\begin{definition}
We can call this equation (same as any other ODEs) a \emph{dynamical system}, and $\bo{x}$ is called the \emph{state} of the dynamical system.  
\end{definition}

\begin{example}
\begin{equation}
    \dot{x} = -3 x^3 - 7 
\end{equation}
\end{example}

\end{flushleft}
\end{frame}




\begin{frame}{State}
%	\framesubtitle{1st order}
	\begin{flushleft}
		
		\emph{State} of a dynamical system is a minimal set of variables that describe the system, in the sense that knowing current state and all future inputs you can predict the behavior of the system.
		
		\begin{example}
			For a spring-damper system, the state variables could be position and velocity of the mass.
		\end{example} 
		\begin{example}
			For a double pendulum, the state variables could be joint angles and joint velocities.
		\end{example} 
		
	\end{flushleft}
\end{frame}






\begin{frame}{ODEs, n-th order}
%\framesubtitle{n-th order}
\begin{flushleft}

The normal form of an \emph{n-th order} ordinary differential equation is:

\begin{equation}
	y^{(n)} = f (y^{(n-1)}, y^{(n-2)}, ...\,, \dot{y}, y, t)
\end{equation}
			
where $y = y(t)$ is the solution of the equation. Same as before, it is a \emph{dynamical system}, but this time we need more variables to describe the state of this system, for example we can use the set $\{ y, \ \dot{y} , ...\,,y^{(n-1)} \}$.

\begin{example}[Pendulum]
\begin{equation}
    \ddot{y} = - 0.1 \dot y - 7\sin(y)
\end{equation}
\end{example}


\begin{example}[DC motor under constant voltage]
\begin{equation}
\begin{cases}
    \dot{y}_1 = - 100 \dot{y}_2 -2 y_1  + 10 \\
    \ddot{y}_2 = -0.1 \dot{y}_2 + 100 y_1
\end{cases}
\end{equation}
\end{example}

\end{flushleft}
\end{frame}




\begin{frame}{Linear ODE, 1st order}
\begin{flushleft}

Linear ODEs of the first order have normal form:

\begin{equation}
    \dot{\bo{x}} = \bo{A} \bo{x}
\end{equation}

\begin{example}
\begin{equation}
\begin{cases}
    \dot{x}_1 = -20 x_1 + 7 x_2 \\
    \dot{x}_2 = 10.5 x_1 - 3 x_2
\end{cases}
\end{equation}
\end{example}

\begin{example}
\begin{equation}
\begin{bmatrix}
\dot{x}_1 \\
\dot{x}_2 \\
\dot{x}_3
\end{bmatrix} 
= 
\begin{bmatrix}
-8   & 5   & 2  \\
 0.5 & -10 & -2 \\
 1   & -1 & -20
\end{bmatrix}
\begin{bmatrix}
x_1 \\
x_2 \\
x_3
\end{bmatrix} 
\end{equation}
\end{example}

\end{flushleft}
\end{frame}




\begin{frame}{Linear differential equations, n-th order}
%\framesubtitle{n-th order}
\begin{flushleft}

A single linear ODE of the n-th order are often written in the form:

\begin{equation}
    a_n y^{(n)} + 
    ... +
    a_2 \ddot{y} + a_1 \dot{y} + 
    a_0 y = 0
\end{equation}

\begin{example}
\begin{equation}
12 \dddot{y} -
    3 \ddot{y} + 5.5 \dot{y} + 
    2 y = 0
\end{equation}
\end{example}

\begin{example}
\begin{equation}
    5 \ddot{y} - 2 \dot{y} + 
    10 y = 0
\end{equation}
\end{example}

\end{flushleft}
\end{frame}




\begin{frame}{ODEs with an input, 1}
	%\framesubtitle{n-th order}
	\begin{flushleft}
		
		Sometimes it is convenient to write an ODE in the form with an \emph{input}, for example:
		
		\begin{equation}
			a_2 \ddot{y} + a_1 \dot{y} + 
			a_0 y = u(t)
		\end{equation}
		
		In this equation $u(t)$ is a function of time. This form offers us many uses:
		
		\begin{itemize}
			\item We can use $u(t)$ to model \emph{control input}, (e.g. voltage, motor torque) that we directly control.
			
			\item We can use $u(t)$ to model external forces acting on the system.
			
			\item We can substitute particular function instead of $u(t)$, e.g. sine wave or step function, to study how the system behaves with such an input.
		\end{itemize}
		
	\end{flushleft}
\end{frame}



\begin{frame}{ODEs with an input, 1}
	\begin{flushleft}
		
		Some examples of linear ODEs with one input:
		
		
		\begin{example}
			\begin{equation}
				\begin{cases}
					\dot{y}_1 = -20 y_1 + 7 y_2 + u \\
					\dot{y}_2 = 10.5 y_1 - 3 y_2
				\end{cases}
			\end{equation}
		\end{example}
		
		\begin{example}
			\begin{equation}
				\begin{bmatrix}
					\dot{x}_1 \\
					\dot{x}_2 \\
					\dot{x}_3
				\end{bmatrix} 
				= 
				\begin{bmatrix}
					-8   & 5   & 2  \\
					0.5 & -10 & -2 \\
					1   & -1 & -20
				\end{bmatrix}
				\begin{bmatrix}
					x_1 \\
					x_2 \\
					x_3
				\end{bmatrix} 
			+
			\begin{bmatrix}
				1    \\
				0    \\
				0   
			\end{bmatrix}
			u
			\end{equation}
		\end{example}
		
	\end{flushleft}
\end{frame}



\begin{frame}{Equations with an input}
	%\framesubtitle{n-th order}
	\begin{flushleft}
		
		General form of an n-th order linear ODE with an input can be presented as follows:
		%
		\begin{equation}
			a_n y^{(n)} + 
			... +
			a_2 \ddot{y} + a_1 \dot{y} + 
			a_0 y = u(t)
		\end{equation}
	
	\bigskip
		
		State-space representation of a linear system with an input is:
		%
		\begin{equation}
			\dot{\bo{x}} = \bo{A} \bo{x} + \bo{B} \bo{u}
		\end{equation}
		
		Note that in latter, $\bo{u}$ can be either scalar or a vector.
		
	\end{flushleft}
\end{frame}







\begin{frame}{Equations with an output}
	%\framesubtitle{n-th order}
	\begin{flushleft}
		
		Equations can also have an output. The meaning of what is an output of an equation depends on the particular use-case - it is not a mathematical issue, it is a question of interpretation. For example, an output can mean:
		
		\begin{itemize}
			\item What we measure (position and orientation of a quadrotor, angular velocity of motor's rotor, etc.).
			
			\item What we care about and/or what we want to control (height of a quadrotor, velocity of a car, etc.)
			
			\item etc.
		\end{itemize}
		
		We often denote output as $y$, and it depends on the state of the system: $y = g(\bo{x})$
		
	\end{flushleft}
\end{frame}


\begin{frame}{Equations with an output}
	%\framesubtitle{n-th order}
	\begin{flushleft}
		
		State-space representation of a linear system with an input and an output is:
		%
		\begin{equation}
		\begin{cases}
				\dot{\bo{x}} = \bo{A} \bo{x} + \bo{B}\bo{u} \\
				\bo{y} = \bo{C}\bo{x}
		\end{cases}
		\end{equation}
		
		If $\bo{u} \in \R$ and $\bo{y} \in \R$ (i.e. if they are scalars) and you want to represent the system with an output as a single ODE, it is typical to treat the output as the ODE variable:
		
		\begin{equation}
			a_n y^{(n)} + 
			... +
			a_2 \ddot{y} + a_1 \dot{y} + 
			a_0 y = u(t)
		\end{equation}
		
	\end{flushleft}
\end{frame}






\begin{frame}{Linear differential equations}
%\framesubtitle{...are what we will study}
\begin{flushleft}

In this course we will focus entirely on linear dynamical systems, expressed as ODEs:

\begin{equation}
    a_n y^{(n)} +
    ... +
    a_2 \ddot{y} + a_1 \dot{y} + 
    a_0 y = u(t)
\end{equation}

or in state-space form:

\begin{equation}
	\begin{cases}
		\dot{\bo{x}} = \bo{A} \bo{x} + \bo{B}\bo{u} \\
		\bo{y} = \bo{C}\bo{x}
	\end{cases}
\end{equation}

If $\bo{u}$ and $\bo{y}$ are scalars, the system is called \emph{single-input single-output (SISO)}, if they are vectors - \emph{multi-input multi-output (MIMO)}.

\bigskip

We can always express a SISO system in either form - ODE or state-space.

\end{flushleft}
\end{frame}




\begin{frame}{ODE to State-Space conversion}
% \framesubtitle{...are what we will study}
\begin{flushleft}

Consider eq. $\dddot{y} + a_2 \ddot{y} + a_1 \dot{y} + a_0 y =u$.

\bigskip

Make a substitution: $x_1 = y$, $x_2 = \dot{y}$, $x_3 = \ddot{y}$. We get:

\begin{align}
        \dot{x}_1 &= \dot{y} = x_2 \\
        \dot{x}_2 &= \ddot{y} = x_3 \\
        \dot{x}_3 &=  u-a_2 \ddot{y} - a_1 \dot{y} - a_0 y = 
        u-a_2 x_3 - a_1 x_2 - a_0 x_1
\end{align}

Which can be directly put in the state-space form:

\begin{equation}
\begin{bmatrix}
\dot{x}_1 \\ \dot{x}_2 \\ \dot{x}_3
\end{bmatrix} 
=
\begin{bmatrix}
0 & 1 & 0 \\ 
0 & 0 & 1 \\
-a_0 & -a_1 & -a_2
\end{bmatrix} 
\begin{bmatrix}
x_1 \\ x_2 \\ x_3
\end{bmatrix} 
+ 
\begin{bmatrix}
0 \\ 0 \\ u
\end{bmatrix}
\end{equation}


\end{flushleft}
\end{frame}




{
\setbeamercolor{background canvas}{bg=mywhitepink}
\begin{frame}{State-Space to ODE conversion, 1}
	% \framesubtitle{...are what we will study}
	\begin{flushleft}
		
		Consider State-Space system:
		%
		\begin{equation}
			\begin{cases}
				\dot{\bo{x}} = \bo{A} \bo{x} \\
				y = \bo{C}\bo{x}
			\end{cases}
		\end{equation}
	
		We want to find an equivalent representation in the ODE form:
		%
		\begin{equation}
			y^{(n)} = d_{n-1} y^{(n-1)} + ... + d_1 \dot y + d_0 y
		\end{equation}
		
		Defining $\bo{d}\T = \begin{bmatrix}
			d_0 & d_1 & ... & d_{n-1}   
		\end{bmatrix}$ and 
	$\bo{y} = \begin{bmatrix}
		 y & \dot y & ... & y^{(n-1)} 
	\end{bmatrix}\T$, we can re-write the ODE as:
		%
		\begin{equation}
		y^{(n)} = \bo{d}\T \bo{y}
		\end{equation}

		Thus, if we can find $\bo{d}$, we can solve the problem.
		
		
	\end{flushleft}
\end{frame}
}



{
	\setbeamercolor{background canvas}{bg=mywhitepink}
\begin{frame}{State-Space to ODE conversion, 2}
	% \framesubtitle{...are what we will study}
	\begin{flushleft}
		
		We can differentiate $y = \bo{C}\bo{x}$ n times:
		%
		\begin{align}
			y &= \bo{C}\bo{x} \\
			\dot y &= \bo{C}\dot{\bo{x}} = \bo{C}\bo{A}\bo{x} \\
			... \\
			y^{(n)} &= \bo{C}\bo{x}^{(n)} = \bo{C}\bo{A}^n\bo{x}
		\end{align}
		
		This gives us relation between $\bo{y}$ and $\bo{x}$:
		%
		\begin{align}
			\bo{y} = 
			\begin{bmatrix}
				\bo{C} \\
				\bo{C} \bo{A} \\
				... \\
				\bo{C}\bo{A}^{n-1}
			\end{bmatrix}
		\bo{x} = \mathcal{O}\bo{x}
		\end{align}
		%
		where matrix $\mathcal{O}$ is called observability matrix.
		
		
	\end{flushleft}
\end{frame}
}


{
	\setbeamercolor{background canvas}{bg=mywhitepink}
\begin{frame}{State-Space to ODE conversion, 3}
	% \framesubtitle{...are what we will study}
	\begin{flushleft}
		
		As long as the observability matrix $\mathcal{O}$ is full rank, we can express the state as: 
		%
		\begin{align}
			\bo{x}=\mathcal{O}^{-1}\bo{y}
		\end{align}
		
		Then we re-write $y^{(n)} = \bo{C}\bo{A}^n\bo{x}$ as:
		%
		\begin{align}
			y^{(n)} = \bo{C}\bo{A}^n\mathcal{O}^{-1}\bo{y}
		\end{align}
		
		Thus, $\bo{d}\T = \bo{C}\bo{A}^n\mathcal{O}^{-1}$ and the ODE takes the form:
		%
		\begin{align}
			y^{(n)} = \bo{C}\bo{A}^n\mathcal{O}^{-1}
			\begin{bmatrix}
				y \\ \dot y \\ ... \\ y^{(n-1)} 
			\end{bmatrix}
		\end{align}
		
		You can see an example in the appendix A.
		
		
	\end{flushleft}
\end{frame}
}


\begin{frame}{Read more}

\begin{itemize}
	
\item 2.14 Analysis and Design of Feedback Control Systems:

\begin{itemize}
	\item  \bref{http://web.mit.edu/2.14/www/Handouts/StateSpace.pdf}{State-Space Representation of LTI Systems}
	
	\item  \bref{http://web.mit.edu/2.14/www/Handouts/StateSpaceResponse.pdf}{Time-Domain Solution of LTI State Equations}
\end{itemize}	
	
\item \bref{https://lpsa.swarthmore.edu/}{Linear Physical Systems Analysis}:

\begin{itemize}
\item State Space Representations of Linear Physical Systems \bref{https://lpsa.swarthmore.edu/Representations/SysRepSS.html}{lpsa.swarthmore.edu/Representations/SysRepSS.html}

\item Transformation: Differential Equation to State Space \bref{https://lpsa.swarthmore.edu/Representations/SysRepTransformations/DE2SS.html}{lpsa.swarthmore.edu/.../DE2SS.html}
\end{itemize}	

\end{itemize}

\end{frame}



\myqrframe



\begin{frame}{Appendix}
	
	\centerline{\huge Appendix A}
	
\end{frame}



\begin{frame}{State Space to ODE conversion, 1}
	\framesubtitle{(extra)}
	\begin{flushleft}
		
		Consider a system in state-space form:
		
		\begin{equation}
			\label{eq:SS}
			\begin{cases}
				\begin{bmatrix}
					\dot x_1 \\ 
					\dot x_2
				\end{bmatrix} 
				= 
				\begin{bmatrix}
					a_{11} & a_{12} \\ 
					a_{21} & a_{22}
				\end{bmatrix} 
				\begin{bmatrix}
					x_1 \\ 
					x_2
				\end{bmatrix} 
				\\
				y = 
				\begin{bmatrix}
					c_1 & 
					c_2
				\end{bmatrix} 
				\begin{bmatrix}
					x_1 \\ 
					x_2
				\end{bmatrix} 
			\end{cases}
			\Longleftrightarrow \ \ 
			\begin{cases}
				\dot{\bo{x}} = \bo{A} \bo{x} \\
				y = \bo{C} \bo{x} 
			\end{cases}
		\end{equation}
		
		We want to rewrite it as a linear ODE:
		
		\begin{equation}
			\label{eq:ODE}
			\ddot{y} + b_2 \dot{y} + b_1 y = 0
		\end{equation}
		
		Note that initial conditions of both equation need to agree.
		
		
	\end{flushleft}
\end{frame}



\begin{frame}{State Space to ODE}
	\begin{flushleft}
		
		Since $y = \bo{C} \bo{x}$, its derivative is $\dot y = \bo{C} \dot{\bo{x}}$:
		
		\begin{equation}
			\dot y = \bo{C} \bo{A} \bo{x}
		\end{equation}
		%
		\begin{equation}
			\dot y = 
			\begin{bmatrix}
				(a_{11}c_1 + a_{21}c_2) & (a_{12}c_1 + a_{22}c_2)
			\end{bmatrix}
		\begin{bmatrix}
			x_1 \\ x_2
		\end{bmatrix}
		\end{equation}
		
		Analogous for $\ddot y$:
		
		\begin{equation}
			\ddot y = \bo{C} \bo{A} \bo{A} \bo{x}
		\end{equation}
		
	\end{flushleft}
\end{frame}




\begin{frame}{State Space to ODE}
	\begin{flushleft}
		
		Combining our results we find the linear transformation between the variables $x_1$, $x_2$ and $y$, $\dot y$:
		
		\begin{equation}
			\begin{bmatrix}
				y \\ \dot y
			\end{bmatrix}
				= 
			\begin{bmatrix}
				c_1                                     & c_2 \\ 
				(a_{11}c_1 + a_{21}c_2) & (a_{12}c_1 + a_{22}c_2)
			\end{bmatrix} 
			\begin{bmatrix}
				x_1 \\ x_2
			\end{bmatrix}
		\end{equation}
		
		Resulting transformation matrix is:
		
		\begin{equation}
			\bo{T} = 
			\begin{bmatrix}
				c_1                                     & c_2 \\ 
				(a_{11}c_1 + a_{21}c_2) & (a_{12}c_1 + a_{22}c_2)
			\end{bmatrix} 
		\end{equation}
		\begin{equation}
			\bo{x}
			= 
			\bo{T}^{-1}
			\begin{bmatrix}
				y \\ 
				\dot y
			\end{bmatrix} 
		\end{equation}
		
	\end{flushleft}
\end{frame}



\begin{frame}{State Space to ODE}
	\begin{flushleft}
		
		Remember that:
		%
		\begin{align}
			\ddot y = \bo{C} \bo{A} \bo{A} \bo{x}
			\\
			\ddot y = \bo{C} \bo{A} \bo{A} \bo{T}^{-1} 
			\begin{bmatrix}
				y \\ 
				\dot y
			\end{bmatrix}
		\end{align}
		
		So, we obtained $\ddot y$ as a linear function of $y$, $\dot y$.       From this it is clear how the same can be generalized to higher dimensions.
		
	\end{flushleft}
\end{frame}



\begin{frame}{State Space to ODE}
	%\framesubtitle{part 5}
	\begin{flushleft}
		
		\textcolor{blue}{\href{https://github.com/SergeiSa/Control-Theory-Slides-Spring-2022/blob/main/ColabNotebooks/StateSpace2ODE.ipynb}{Check out the code implementation.}}
		
		\bigskip
		
		
		\centerline{\textcolor{black}{\qrcode[height=2.1in]{https://github.com/SergeiSa/Control-Theory-Slides-Spring-2022/blob/main/ColabNotebooks/StateSpace2ODE.ipynb}}}
		
		
	\end{flushleft}
\end{frame}


\end{document}
